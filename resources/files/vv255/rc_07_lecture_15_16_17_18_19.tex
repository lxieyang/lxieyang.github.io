\documentclass[10pt]{beamer}
\usetheme{default}
\usepackage{float}
\usepackage{animate}
\usepackage{amsmath}

\usepackage{esvect}
%\usepackage[T1]{fontenc}

%IndentfFirst 
%\usepackage{indentfirst}
%\setlength{\parindent}{2em}
%\setlength{\parskip}{2em}


%Geometry
\usepackage{geometry}
\geometry{left = 0.25in,right = 0.25in}



%Adjust frametitle position (height,lateral position, etc)
\defbeamertemplate*{frametitle}{smoothbars theme}
{%
	%\nointerlineskip%
	\begin{beamercolorbox}[wd=\paperwidth,leftskip=.5cm,rightskip=.3cm plus1fil,vmode]{frametitle}
		\vskip +4.5ex
		\usebeamerfont*{frametitle}\insertframetitle%
		\vskip -0.9ex
	\end{beamercolorbox}%
}


\begin{document}


%headline

\setbeamertemplate{headline}{
\parbox{\linewidth}{\vspace*{8pt}\centering{ 
		    {\color{blue!40!black}\insertsection}
		}
	}
}


%footline
\setbeamertemplate{footline}[text line]{%
	\color{blue!40!black}\parbox{\linewidth}{\vspace*{-8pt}Michael Liu ~ (\insertshortinstitute)\hfill\insertshorttitle\hfill\insertshortdate~~~~~~\insertframenumber{}~/~\inserttotalframenumber}}

%Remove navigation symbols
\setbeamertemplate{navigation symbols}{}

\setbeamertemplate{frametitle continuation}[from second] 

\newcommand{\tabincell}[2]{\begin{tabular}{@{}#1@{}}#2\end{tabular}}

%\let\oldframe\frame\renewcommand\frame[1][allowframebreaks]{\oldframe[#1]}

%Title page	
\title[Vv255 Applied Calculus III]{Vv255 Applied Calculus III\\{\small Recitation VII}}   
\author[Michael Liu]{LIU Xieyang\\{\tiny Teaching Assistant}} 
\institute[UM-SJTU JI]{University of Michigan - Shanghai Jiaotong University \\Joint Institute}
\date[Summer 2015]{Summer Term 2015} 
\begin{frame}
	\titlepage
\end{frame}

%Table of Contents (All)
%\begin{frame}
%	\frametitle{Table of Contents}
%	\tableofcontents
%\end{frame}

%Table of contents (before, highlight each section)
\AtBeginSection[]{
	\begin{frame}
		\frametitle{Table of Contents}
		\frametitle{Contents}
		\tableofcontents[currentsection]
	\end{frame}}




%section
\section{Lecture 15: Double Integral} 



\begin{frame}[allowframebreaks]{Double Integral}
	\begin{figure}[H]
		\centering
		\includegraphics[scale = 0.26]{doublepartition}
	\end{figure}
	Net signed volume:
	$$\iint\limits_{D} f(x,y) dA = \lim\limits_{n\rightarrow \infty}\sum_{i}^{n}f(x_i^*,y_i^*)\Delta A_i$$
A function $f (x, y)$ is called {\color{red} integrable} if the limit actually exists and that its
value does not depend on the choice of the partition.
\end{frame}

\begin{frame}[allowframebreaks]{Fubini's Theorem}
	\begin{figure}[H]
		\centering
		\includegraphics[scale = 0.45]{fubini}
	\end{figure}
	When $f (x, y)$ can be factored as the product of a function of $x$ only and a
	function of $y$ only, the double integral of $f$ can be written in a simple form.
	$$\iint\limits_{D} f(x,y) dA  = \int_{a}^{b} g(x) dx \int_{c}^{d}h(y)dy$$

\end{frame}

\begin{frame}[allowframebreaks]{Two Basic Types of Double Integral}
	\begin{figure}[H]
		\centering
		\includegraphics[scale = 0.45]{typeiii}
		\caption{Type I ~~~~~~~~~~~~~~~~~~~~~~~~~~~~~~~~~~~~~~~~~~~~~~~~~~~~Type II}
	\end{figure}
	$$\text{Type I: } ~~~ \iint\limits_D f(x,y)dA = \int_{a}^{b}\int_{g_1(x)}^{g_2(x)}f(x,y)dydx$$
	$$\text{Type II: } ~~~ \iint\limits_D f(x,y)dA = \int_{c}^{d}\int_{h_1(x)}^{h_2(x)}f(x,y)dxdy$$

	
\end{frame}







\section{Lecture 16: Applications of Double integral}

\begin{frame}[allowframebreaks]{Double intergral in polar coordinates}
\begin{figure}[H]
	\centering
	\includegraphics[scale = 0.3]{polar}
\end{figure}
$$\iint\limits_R f(x,y)dA = \int_{\alpha}^{\beta}\int_{a}^{b}f(r\cos\theta,r\sin\theta)~{\color{red}\mathbf{r}}dr~d\theta$$
\centering
{$\color{red}\mathbf{r}$} is known as the {\color{red}metric coefficient}.
	
\end{frame}

\begin{frame}[allowframebreaks]{Applications of Double integral}
\begin{enumerate}
	\item Volume \& area.
	\item Average value
	$$\text{Average value} ~= \dfrac{1}{A} \iint\limits_D f(x,y) dA$$
	\item Total mass of a lamina (inhomogeneous)
	$$m = \iint\limits_D \rho (x,y) dA$$
	\item Center of mass/Center of gravity
	$$\bar{x} = \dfrac{M_y}{m} = \dfrac{\iint\limits_D x\rho(x,y)dA}{\iint\limits_D\rho(x,y)dA} ~~~~~ \bar{y} = \dfrac{M_x}{m} = \dfrac{\iint\limits_D y\rho(x,y)dA}{\iint\limits_D\rho(x,y)dA}$$
	where $M_y$ and $M_x$ are the moments w.r.t. to the $y$ and $x$ axes.
	\item Centroid (the center of mass of a homogeneous lamina):
	$$\bar{x} = \dfrac{\iint\limits_DxdA}{\iint\limits_DdA} = \dfrac{1}{\text{Area}} \iint\limits_DxdA ~~~~~ \bar{y} = \dfrac{\iint\limits_DydA}{\iint\limits_DdA} = \dfrac{1}{\text{Area}} \iint\limits_DydA$$
	\item Surface area:
	\begin{figure}[H]
		\centering
		\includegraphics[scale = 0.33]{surface1} ~~~\includegraphics[scale = 0.33]{surface2}
	\end{figure}
		\begin{figure}[H]
			\centering
			\includegraphics[scale = 0.33]{surface1} ~~~\includegraphics[scale = 0.33]{surface2}
		\end{figure}
		$$A(S) = \lim\limits_{m,n\rightarrow \infty} \sum_{i = 1}^{m}\sum_{j = 1}^{n}\Delta T_{ij}\lim\limits_{m,n\rightarrow \infty} \sum_{i = 1}^{m}\sum_{j = 1}^{n}\Delta ~|\mathbf{a}\times\mathbf{b}|$$
		$$\includegraphics[scale = 0.3]{ab}$$
			\begin{figure}[H]
				\centering
				\includegraphics[scale = 0.2]{surface1} ~~~\includegraphics[scale = 0.2]{surface2}
			\end{figure}
			$$\includegraphics[scale = 0.28]{tij}$$
			$$\text{Therefore, }~~~A(S) = \iint\limits_D\sqrt{[f_x(x,y)]^2 + [f_y(x,y)]^2 + 1}~dA$$
\end{enumerate}

More advanced:
\begin{enumerate}
	\item Electric potential $V$
	$$V = \dfrac{1}{4\pi\epsilon_0}\iint\limits_s\dfrac{\rho_s}{R}ds$$
	\item $\cdots$
\end{enumerate}
	
	
\end{frame}












\section{Lecture 17: Triple integral}

\begin{frame}[allowframebreaks]{Triple integral \& Fubini's Theorem}
	Triple integral:
	$$\iiint\limits_Bf(x,y,z)dV = \lim\limits_{I,m,n\rightarrow\infty}\sum_{i = 1}^{I}\sum_{j = 1}^{m}\sum_{k = 1}^{n}f(x_{ijk}^*,y_{ijk}^*,z_{ijk}^*)\Delta V$$
	\centering
	Find the boundary expression!
	$$$$
	$$\includegraphics[scale = 0.45]{fubini3}$$
\centering
There are five other possible orders, all of which give the same value.
\end{frame}








\section{Lecture 18: Triple integral in Cylindrical and Spherical coordinates}

\subsection{Intro. to Coordinate Systems}
\begin{frame}[allowframebreaks]{Orthogonal Coordinate Systems}
	\begin{itemize}
		\item A tool to ease the solving process for problems
		with certain geometry;
		\item {\color{blue}A point in 3D space can be determined by the
		intersection of three surfaces.}
	\item If these three surfaces are {\color{red}mutually orthogonal} to
	each other, this sets up an orthogonal coordinate
	system.
	$$\includegraphics[scale = 0.23]{coordinates}$$
	$$\text{\small Point in space is an
		intersection of
		coordinate surfaces.}$$$$\text{\small To describe location of each
		point in space we need 3
		sets of surfaces.}$$
	\item {\color{red} Unit vectors} are defined as the normal direction of each
	coordinate surface.
	\end{itemize}
\end{frame}


\begin{frame}[allowframebreaks]{Cartesian Coordinates}
$$\includegraphics[scale = 0.4]{cartesian}$$
$$(x,y,z)$$
Coordinate system where all coordinate surfaces are planar is
called {\color{red}Cartesian}. Note that in this system, the direction of
coordinate vectors is the same at every point in space
\end{frame}

\begin{frame}[allowframebreaks]{Cylindrical Coordinates}
	$$\includegraphics[scale = 0.4]{cy}$$
	$$(\rho~(\text{or }~r),\phi~(\text{or}~\theta),z)$$
In a cylindrical coordinate
system, a cylindrical
surface and two planar
surfaces all orthogonal to
each other define location
of a point in space.
\\The
coordinate vectors are $\hat{\rho}~(\text{or}~\hat{r}), \hat{\phi}~(\text{or}~\hat{\theta}), \hat{z}$. Note, in this system the
direction of coordinate
vectors changes from point
to point.
\newpage
$$\includegraphics[width = 2.3in]{cy2}\includegraphics[width = 2.3in, height = 1.9in]{cy3}$$
\end{frame}


\begin{frame}[allowframebreaks]{Spherical Coordinates}
	$$\includegraphics[scale = 0.4]{sp}$$
	$$(R~(\text{or }~\rho),\theta~(\text{or}~\phi),\phi~(\text{or} ~ \theta))$$
	In a spherical coordinate
	system the surfaces of
	spheres, cones and planes are
	coordinate surfaces. The
	corresponding coordinate
	vectors $(\hat{R}~(\text{or }~\hat{\rho}),\hat{\theta}~(\text{or}~\hat{\phi}),\hat{\phi}~(\text{or} ~ \hat{\theta}))$ also change their
	direction from point to point.
	\newpage
	$$\includegraphics[width = 2.3in]{sp2}\includegraphics[width = 2.3in, height = 2.0in]{sp3}$$
	
\end{frame}

\subsection{Triple integral in Cylindrical and Spherical coordinates}
\begin{frame}[allowframebreaks]{Triple integral in Cylindrical and Spherical coordinates}
Cylindrical:
$$\iiint\limits_E f(r,\theta,z) ~dr~rd\theta ~dz = \iiint\limits_E f(r,\theta,z) \underline{r}drd\theta dz$$
Spherical:
$$\iiint\limits_E f(\rho,\phi,\theta)~d\rho ~ \rho d\phi ~ \rho\sin\phi d\theta = \iiint\limits_E f(\rho,\phi,\theta)\underline{\rho^2\sin\phi} d\rho d\theta d\phi$$
The additional $r$ and $\rho^2\sin\phi$ can also be interpreted as {\color{red}Jacobian}.
\end{frame}





\section{Lecture 19: Jacobian}


\begin{frame}[allowframebreaks]{Jacobian}
Definition:\\
\includegraphics[scale = 0.35]{j21}
\\~\\Theorem:\\
\includegraphics[scale = 0.35]{j22}
\end{frame}


\begin{frame}[allowframebreaks]{Jacobian}
	Definition:\\
	\includegraphics[scale = 0.39]{j31}
\end{frame}

\end{document}