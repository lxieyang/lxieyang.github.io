\documentclass[10pt]{beamer}
\usetheme{default}
\usepackage{float}
\usepackage{animate}
\usepackage{amsmath}

\usepackage{esvect}
%\usepackage[T1]{fontenc}

%IndentfFirst 
%\usepackage{indentfirst}
%\setlength{\parindent}{2em}
%\setlength{\parskip}{2em}


%Geometry
\usepackage{geometry}
\geometry{left = 0.25in,right = 0.25in}



%Adjust frametitle position (height,lateral position, etc)
\defbeamertemplate*{frametitle}{smoothbars theme}
{%
	%\nointerlineskip%
	\begin{beamercolorbox}[wd=\paperwidth,leftskip=.5cm,rightskip=.3cm plus1fil,vmode]{frametitle}
		\vskip +4.5ex
		\usebeamerfont*{frametitle}\insertframetitle%
		\vskip -0.9ex
	\end{beamercolorbox}%
}


\begin{document}


%headline

\setbeamertemplate{headline}{
\parbox{\linewidth}{\vspace*{8pt}\centering{ 
		    {\color{blue!40!black}\insertsection}
		}
	}
}


%footline
\setbeamertemplate{footline}[text line]{%
	\color{blue!40!black}\parbox{\linewidth}{\vspace*{-8pt}Michael Liu ~ (\insertshortinstitute)\hfill\insertshorttitle\hfill\insertshortdate~~~~~~\insertframenumber{}~/~\inserttotalframenumber}}

%Remove navigation symbols
\setbeamertemplate{navigation symbols}{}

\setbeamertemplate{frametitle continuation}[from second] 

\newcommand{\tabincell}[2]{\begin{tabular}{@{}#1@{}}#2\end{tabular}}

%\let\oldframe\frame\renewcommand\frame[1][allowframebreaks]{\oldframe[#1]}

%Title page	
\title[Vv255 Applied Calculus III]{Vv255 Applied Calculus III\\{\small Recitation V}}   
\author[Michael Liu]{LIU Xieyang\\{\tiny Teaching Assistant}} 
\institute[UM-SJTU JI]{University of Michigan - Shanghai Jiaotong University \\Joint Institute}
\date[Summer 2015]{Summer Term 2015} 
\begin{frame}
	\titlepage
\end{frame}

%Table of Contents (All)
%\begin{frame}
%	\frametitle{Table of Contents}
%	\tableofcontents
%\end{frame}

%Table of contents (before, highlight each section)
\AtBeginSection[]{
	\begin{frame}
		\frametitle{Table of Contents}
		\frametitle{Contents}
		\tableofcontents[currentsection]
	\end{frame}}




%section
\section{Lecture 10: The Chain Rule} 



\begin{frame}[allowframebreaks]{Chain Rule}
Version 1:
\begin{enumerate}
	\item Suppose $f (x, y)$ is differentiable, where
	\item $x = x(t)$ and $y = y(t)$ are differentiable 	functions of $t$,
	\\then the composite function $f (x(t), y(t))$	is a differentiable function of $t$ and
	$$\dfrac{df}{dt} = \dfrac{\partial f}{\partial x}\dfrac{dx}{dt} + \dfrac{\partial f}{\partial y}\dfrac{dy}{dt}$$
\end{enumerate}

Version 2:
\begin{enumerate}
	\item Suppose $z = f (x, y)$ and $y = g(x)$, then $x$ is the only independent variable
	\begin{align*}
	\dfrac{dz}{dx} &= \dfrac{\partial z}{\partial x}\dfrac{dx}{dx} + \dfrac{\partial z}{\partial y}\dfrac{dy}{dx}
	\\&= \dfrac{\partial z}{\partial x} + \dfrac{\partial z}{\partial y}\dfrac{dy}{dx}
	\end{align*}
\end{enumerate}

\newpage
Version 3:
\begin{enumerate}
	\item If $f (x, y, z)$, $x(r, s)$, $y(r, s)$, and $z(r, s)$ are differentiable, then $f$ has partial
	derivatives with respect to $r$ and $s$, given by
	$$\dfrac{\partial f}{\partial r} = \dfrac{\partial f}{\partial x} \dfrac{\partial x}{\partial r} + \dfrac{\partial f}{\partial y} \dfrac{\partial y}{\partial r} + \dfrac{\partial f}{\partial z} \dfrac{\partial z}{\partial r}$$
		$$\dfrac{\partial f}{\partial s} = \dfrac{\partial f}{\partial x} \dfrac{\partial x}{\partial s} + \dfrac{\partial f}{\partial y} \dfrac{\partial y}{\partial s} + \dfrac{\partial f}{\partial z} \dfrac{\partial z}{\partial s}$$
\end{enumerate}


\end{frame}

\begin{frame}[allowframebreaks]{Implicit Differentiation}
Version 1:\\
Suppose that $F(x, y)$ is differentiable and that the equation $F(x, y) = 0$ defines $y$
as a differentiable function of $x$. Then at any point where $F_y\neq 0$,
$$\dfrac{dy}{dx} = -\dfrac{F_x}{F_y}$$



Version 2:\\
Suppose $F$ is a function of $x$, $y$ and $z$, if the following conditions are satisfied
\begin{enumerate}
	\item The partial derivatives $F_x$ , $F_y$ , and $F_z$ are continuous throughout an open
	region $R$ in space containing the point $(x_0, y_0, z_0)$.
	\item For some constant $c$, $F(x_0,y_0,z_0) = c$, $F_z(x_0,y_0,z_0) \neq 0$,
	\\then the equation $F(x, y, z) = c$ defines $z$ implicitly as a differentiable function
	of $x$ and $y$ near $(x_0, y_0, z_0)$, and the partial derivatives of $z$ are given by
	$$\dfrac{\partial z}{\partial x} = -\dfrac{F_x}{F_z}~~~~~~\dfrac{\partial z}{\partial y} = - \dfrac{F_y}{F_z}$$
\end{enumerate}	
\end{frame}











%section
\section{Lecture 11: Directional Derivatives and Gradient Vector} 



\begin{frame}[allowframebreaks]{Directional Derivatives \& Gradient Vector}
	
\begin{figure}[H]
	\centering
	\includegraphics[scale = 0.35]{gradient}
	\caption{Directional derivatives \& Gradient Vector}
\end{figure}
	
	
\end{frame}

\begin{frame}[allowframebreaks]{Directional Derivatives }
	
	\begin{figure}[H]
		\centering
		\includegraphics[scale = 0.25]{gradient}
	\end{figure}
\noindent 
Let us consider a scalar function of space coordinates $V(x, y , z)$, which may
represent, say, the temperature distribution in a building, the altitude of a mountainous
terrain, or the electric potential in a region. The magnitude of $V$, in general,
depends on the position of the point in space, but it may be constant along certain
lines or surfaces.
\\The figure above shows two surfaces on which the magnitude of $V$ is
constant and has the values $V_1$ and $V_1 + dV$, respectively, where $dV$ indicates a small
change in $V$.


\newpage
	\begin{figure}[H]
		\centering
		\includegraphics[scale = 0.25]{gradient}
	\end{figure}
	\noindent 

Point $P_1$ is on surface $V_1$;
$P_2$ is the corresponding point on surface $V_1 + dV$ along the normal vector $d\mathbf{n}$; and
$P_3$ is a point close to $P_2$ along another vector $d\mathbf{l}\neq d\mathbf{n}$.\\
For the same change $dV$ in
$V$, the space rate of change, $dV/ dl$, is obviously greatest along $d\mathbf{n}$ because $dn$ is the shortest distance between the two surfaces.
\\Since the magnitude of $dV/dl$ depends
on the direction of $d\mathbf{l}$, $dV/dl$ is a {\color{red}directional derivative}.
\newpage
{\bf We define the vector that
	represents both the magnitude and the direction of the maximum space rate of increase
	of a scalar as the {\color{red}gradient} of that scalar.} We write:
$$\textbf{grad}V \equiv \mathbf{a}_n\dfrac{dV}{dn}$$
For brevity it is customary to employ the operator {\it del}, represented by the symbol
$\nabla $ and write $\nabla V$ in place of $\textbf{grad}V$. Thus,
$$\nabla V\equiv \mathbf{a}_n\dfrac{dV}{dn}$$
We have assumed that $dV$ is positive (an increase in $V$); if $dV$ is negative (a decrease in $V$ from $P_1$ to $P_2$), $\nabla V$ will be negative in the $\mathbf{a}_n$ direction.~~~~~~~~~~~~~~~~~~~~~~~~~~~~~~~~~~~~~~~~~~~~~~~~~~\\~\\~\\~
\\The directional derivative along $d\mathbf{l}$ is

\begin{align*}
\dfrac{dV}{dl} &= \dfrac{dV}{dn}\dfrac{dn}{dl}
\\&= \dfrac{dV}{dn}\cos\alpha
\\&= \dfrac{dV}{dn}\mathbf{a}_n\cdot\mathbf{a}_l = (\nabla V)\cdot \mathbf{a}_l
\end{align*}
The above equation states that the space rate of increase of $V$ in the $\mathbf{a}_l$, direction is equal
to the projection (the component) of the gradient of $V$ in that direction.


\end{frame}



\begin{frame}[allowframebreaks]{ Gradient Vector}
	\begin{align*}
	\dfrac{dV}{dl} = \dfrac{dV}{dn}\mathbf{a}_n\cdot\mathbf{a}_l = (\nabla V)\cdot \mathbf{a}_l
	\end{align*}

We can also write the above equation as:
$$dV = (\nabla V)\cdot d\mathbf{l}$$
where $dl = \mathbf{a}_ldl$. 
\\Now, note that $dV$ is the total differential of $V$ as a result of
a change in position; it can be expressed in terms of
the differential changes in coordinates:
$$dV = \dfrac{\partial V}{\partial x}dx + \dfrac{\partial V}{\partial y}dy + \dfrac{\partial V}{\partial z}dz$$
Note that $$d\mathbf{l} = \mathbf{a}_xdx + \mathbf{a}_ydy + \mathbf{a}_zdz$$
Then, expressing $dV$ as the dot product of two vectors, we have:
\begin{align*}
dV &=  \left(\mathbf{a}_x\dfrac{\partial V}{\partial x} + \mathbf{a}_y\dfrac{\partial V}{\partial y} + \mathbf{a}_z\dfrac{\partial V}{\partial z}\right)  \cdot\left(\mathbf{a}_xdx + \mathbf{a}_ydy + \mathbf{a}_zdz\right)
\\&=\left(\mathbf{a}_x\dfrac{\partial V}{\partial x} + \mathbf{a}_y\dfrac{\partial V}{\partial y} + \mathbf{a}_z\dfrac{\partial V}{\partial z}\right) \cdot d\mathbf{l}
\end{align*}
With a little comparison, we could get that 
$$\nabla V = \mathbf{a}_x\dfrac{\partial V}{\partial x} + \mathbf{a}_y\dfrac{\partial V}{\partial y} + \mathbf{a}_z\dfrac{\partial V}{\partial z}$$
or
$$\nabla V = \left(\mathbf{a}_x\dfrac{\partial }{\partial x} + \mathbf{a}_y\dfrac{\partial }{\partial y} + \mathbf{a}_z\dfrac{\partial }{\partial z}\right) V$$
In view of the equation above, it is convenient to consider $V$ in Cartesian coordinates as a
vector differential {\bfseries\color{red}operator}
$$\nabla \equiv \mathbf{a}_x\dfrac{\partial }{\partial x} + \mathbf{a}_y\dfrac{\partial }{\partial y} + \mathbf{a}_z\dfrac{\partial }{\partial z}$$
\end{frame}


\begin{frame}[allowframebreaks]{ Algebra Rules for Gradients}
Algebra Rules for Gradients
\begin{figure}[H]
	\centering
	\includegraphics[scale = 0.45]{rule}
\end{figure}

\end{frame}

\begin{frame}[allowframebreaks]{ More about $\nabla$ Operator (Will be covered later in this semester)}
$\nabla$ Operator:
$$\nabla \equiv \mathbf{a}_x\dfrac{\partial }{\partial x} + \mathbf{a}_y\dfrac{\partial }{\partial y} + \mathbf{a}_z\dfrac{\partial }{\partial z}$$
Gradient:
$$\nabla V = \left(\mathbf{a}_x\dfrac{\partial }{\partial x} + \mathbf{a}_y\dfrac{\partial }{\partial y} + \mathbf{a}_z\dfrac{\partial }{\partial z}\right) V$$

Divergence:
$$\text{div}~\mathbf{A} = \nabla\cdot \mathbf{A} = \dfrac{\partial A_x}{\partial x} + \dfrac{\partial A_y}{\partial y} + \dfrac{\partial A_z}{\partial z} $$

Curl (Rotation):

\begin{align*}
\text{curl} ~\mathbf{A} = \nabla\times\mathbf{A} =\left\vert
\begin{matrix}
\mathbf{a}_x & \mathbf{a}_y & \mathbf{a}_z\\
\dfrac{\partial }{\partial x} & \dfrac{\partial }{\partial y} & \dfrac{\partial }{\partial z}\\
A_x & A_y & A_z
\end{matrix}
\right\vert
\end{align*}

\end{frame}



\end{document}