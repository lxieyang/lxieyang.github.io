\documentclass{beamer}
\usetheme{default}
\usepackage{float}
\usepackage{animate}
%\usepackage[T1]{fontenc}

%IndentfFirst 
%\usepackage{indentfirst}
%\setlength{\parindent}{2em}
%\setlength{\parskip}{2em}


%Geometry
\usepackage{geometry}
\geometry{left = 0.25in,right = 0.25in}



%Adjust frametitle position (height,lateral position, etc)
\defbeamertemplate*{frametitle}{smoothbars theme}
{%
	%\nointerlineskip%
	\begin{beamercolorbox}[wd=\paperwidth,leftskip=.5cm,rightskip=.3cm plus1fil,vmode]{frametitle}
		\vskip +4.5ex
		\usebeamerfont*{frametitle}\insertframetitle%
		\vskip -0.9ex
	\end{beamercolorbox}%
}


\begin{document}


%headline

\setbeamertemplate{headline}{
\parbox{\linewidth}{\vspace*{8pt}\centering{ 
		    {\color{blue!40!black}\insertsection}
		}
	}
}


%footline
\setbeamertemplate{footline}[text line]{%
	\color{blue!40!black}\parbox{\linewidth}{\vspace*{-8pt}Michael Liu ~ (\insertshortinstitute)~~~~~~~~~~~~~~~~~~~~~~~~\insertshorttitle\hfill\insertshortdate~~~~~~\insertframenumber{}~/~\inserttotalframenumber}}

%Remove navigation symbols
\setbeamertemplate{navigation symbols}{}

\setbeamertemplate{frametitle continuation}[from second] 

\newcommand{\tabincell}[2]{\begin{tabular}{@{}#1@{}}#2\end{tabular}}

%\let\oldframe\frame\renewcommand\frame[1][allowframebreaks]{\oldframe[#1]}

%Title page	
\title[Vv255 Applied Calculus III]{Vv255 Applied Calculus III\\{\small Recitation II}}   
\author[Michael Liu]{LIU Xieyang\\{\tiny Teaching Assistant}} 
\institute[UM-SJTU JI]{University of Michigan - Shanghai Jiaotong University \\Joint Institute}
\date[Summer 2015]{Summer Term 2015} 
\begin{frame}
	\titlepage
\end{frame}

%Table of Contents (All)
\begin{frame}
	\frametitle{Table of Contents}
	\tableofcontents
\end{frame}

%Table of contents (before, highlight each section)
%\AtBeginSection[]{
%	\begin{frame}
		%\frametitle{Table of Contents}
%		\frametitle{Contents}
%		\tableofcontents[currentsection]
%	\end{frame}




%section
\section{Lecture 4: Parametric Equations, Equations of lines and planes} 



\begin{frame}[allowframebreaks]{Multivariable calculus}
	Multivariable calculus (A.K.A. multivariate calculus) is the extension of calculus
	in one variable to calculus in more than one variable.
\begin{itemize}
	\item Pay attention to the figures on Page 2, Lecture 1
\end{itemize}
\begin{enumerate}
	\item Different \alert{dimensions} will lead to different meanings/interpretations of the \alert{same} equation.
	\begin{itemize}
		\item For example $x^2 + y^2 = 1$ in $\mathbb{R}^2$ and $\mathbb{R}^3$.
	\end{itemize}
	
	\item In 2D analytical geometry, the graph of an equation involving $x$ and $y$ is a \alert{curve} in $\mathbb{R}^2$. \\In 3D analytical geometry, an equation in $x$, $y$, and $z$ represents a \alert{surface} in $\mathbb{R}^3$.
	
	\item Distance formula in $\mathbb{R}^3$: $|P_1P_2| = \sqrt{(x_2 - x_1)^2 + (y_2 - y_1)^2 + (z_2-z_1)^2}$
\end{enumerate}


\end{frame}









\section{Lecture 5:}


\begin{frame}[allowframebreaks]{Multivariable calculus}
	Multivariable calculus (A.K.A. multivariate calculus) is the extension of calculus
	in one variable to calculus in more than one variable.
	\begin{itemize}
		\item Pay attention to the figures on Page 2, Lecture 1
	\end{itemize}
	\begin{enumerate}
		\item Different \alert{dimensions} will lead to different meanings/interpretations of the \alert{same} equation.
		\begin{itemize}
			\item For example $x^2 + y^2 = 1$ in $\mathbb{R}^2$ and $\mathbb{R}^3$.
		\end{itemize}
		
		\item In 2D analytical geometry, the graph of an equation involving $x$ and $y$ is a \alert{curve} in $\mathbb{R}^2$. \\In 3D analytical geometry, an equation in $x$, $y$, and $z$ represents a \alert{surface} in $\mathbb{R}^3$.
		
		\item Distance formula in $\mathbb{R}^3$: $|P_1P_2| = \sqrt{(x_2 - x_1)^2 + (y_2 - y_1)^2 + (z_2-z_1)^2}$
	\end{enumerate}
	
	
\end{frame}


\end{document}