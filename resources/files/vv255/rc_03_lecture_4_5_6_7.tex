\documentclass[10pt]{beamer}
\usetheme{default}
\usepackage{float}
\usepackage{animate}

\usepackage{esvect}
%\usepackage[T1]{fontenc}

%IndentfFirst 
%\usepackage{indentfirst}
%\setlength{\parindent}{2em}
%\setlength{\parskip}{2em}


%Geometry
\usepackage{geometry}
\geometry{left = 0.25in,right = 0.25in}



%Adjust frametitle position (height,lateral position, etc)
\defbeamertemplate*{frametitle}{smoothbars theme}
{%
	%\nointerlineskip%
	\begin{beamercolorbox}[wd=\paperwidth,leftskip=.5cm,rightskip=.3cm plus1fil,vmode]{frametitle}
		\vskip +4.5ex
		\usebeamerfont*{frametitle}\insertframetitle%
		\vskip -0.9ex
	\end{beamercolorbox}%
}


\begin{document}


%headline

\setbeamertemplate{headline}{
\parbox{\linewidth}{\vspace*{8pt}\centering{ 
		    {\color{blue!40!black}\insertsection}
		}
	}
}


%footline
\setbeamertemplate{footline}[text line]{%
	\color{blue!40!black}\parbox{\linewidth}{\vspace*{-8pt}Michael Liu ~ (\insertshortinstitute)\hfill\insertshorttitle\hfill\insertshortdate~~~~~~\insertframenumber{}~/~\inserttotalframenumber}}

%Remove navigation symbols
\setbeamertemplate{navigation symbols}{}

\setbeamertemplate{frametitle continuation}[from second] 

\newcommand{\tabincell}[2]{\begin{tabular}{@{}#1@{}}#2\end{tabular}}

%\let\oldframe\frame\renewcommand\frame[1][allowframebreaks]{\oldframe[#1]}

%Title page	
\title[Vv255 Applied Calculus III]{Vv255 Applied Calculus III\\{\small Recitation II}}   
\author[Michael Liu]{LIU Xieyang\\{\tiny Teaching Assistant}} 
\institute[UM-SJTU JI]{University of Michigan - Shanghai Jiaotong University \\Joint Institute}
\date[Summer 2015]{Summer Term 2015} 
\begin{frame}
	\titlepage
\end{frame}

%Table of Contents (All)
%\begin{frame}
%	\frametitle{Table of Contents}
%	\tableofcontents
%\end{frame}

%Table of contents (before, highlight each section)
\AtBeginSection[]{
	\begin{frame}
		\frametitle{Table of Contents}
		\frametitle{Contents}
		\tableofcontents[currentsection]
	\end{frame}}




%section
\section{Lecture 4: Parametric Equations, Equations of lines and planes} 



\begin{frame}[allowframebreaks]{Line in $\mathbb{R}^3$}
	
\begin{itemize}
	\item Vector equation:
	$$\mathbf{r} = \mathbf{r}_0 + t\mathbf{v}$$
	
	\item Parametric equations:
	$$x = x_0 + at ~~~ y = y_0 + bt ~~~ z = z_0 + ct$$
	
	\item Symmetric equations:
	$$\dfrac{x- x_0}{a} = \dfrac{y - y_0}{b} = \dfrac{z - z_0 }{c}$$
	
	\item Line segments from $\mathbf{r}_1$ to $\mathbf{r}_2$ is given by:
	$$\mathbf{r} = (1-t)\mathbf{r}_1 + t\mathbf{r}_2~~~~~0\leq t\leq 1$$

\end{itemize}

\end{frame}



\begin{frame}[allowframebreaks]{Plane in $\mathbb{R}^3$}

	\begin{itemize}
		\item Vector equation:
		$$\mathbf{n}\cdot(\mathbf{r}- \mathbf{r}_0) = 0$$
		
		\item Scalar equation:
		$$a(x - x_0) + b(y - y_0) + c(z - z_0) = 0$$
		
		\item Linear equation:
		$$ax + by + cz + d = 0 ~~~ \text{where} ~~~ d = -ax_0 - by_0 - cz_0$$
		
		\item Distance $D$ from a Point to a Line in Space:
		$$D = \dfrac{\left\vert\overrightarrow{P_0P_1}\times \mathbf{v}\right\vert}{|\mathbf{v}|}$$
		
		\item Distance $D$ from a Point to a Plane:
		$$D = \dfrac{\left\vert\overrightarrow{P_0P_1}\cdot\mathbf{n}\right\vert}{|\mathbf{n}|}$$
		
	\end{itemize}

\end{frame}







\section{Lecture 5: Vector-valued functions; Derivatives and Integrals}

\begin{frame}[allowframebreaks]{Limit, Continuity, Derivative, Integral}
	
	\begin{itemize}
		\item Limit (A practical way to calculate)
		\begin{figure}[H]
			\centering
			\includegraphics[scale = 0.18]{limit}
		\end{figure}
	
	\item Continuity:
	\\A vector function $\mathbf{r}(t)$ is continuous at a point $t = a$ in its domain if
	$$\lim_{t\rightarrow a}\mathbf{r}(t) = \mathbf{r}(a)$$
	The function is {\color{red}continuous} if it is {\color{red}continuous} at every point in its domain.

	\item Derivative:
	\begin{figure}[H]
		\centering
		\includegraphics[scale = 0.24]{de}
	\end{figure}
	\noindent $\mathbf{r}^\prime(t)$ is known as the {\color{red}tangent vector}.
	
	\item Differentiation rules for vector-valued function.
	\begin{figure}[H]
		\centering
		\includegraphics[scale = 0.85]{rule}
	\end{figure}
	
	\item Indefinite integral
	$$\int\mathbf{r}(t)dt = \mathbf{R}(t) + \mathbf{C}$$
	
	\item Definite integral
	\begin{figure}[H]
		\centering
	\includegraphics[scale = 0.25]{def}
	\end{figure}
	
	\item The Fundamental Theorem of Calculus
	\begin{figure}[H]
		\centering
		\includegraphics[scale = 0.25]{ftc}
	\end{figure}
	
	\end{itemize}
	
\end{frame}









\section{Lecture 6: Arc Length and Curvature}

\begin{frame}[allowframebreaks]{$\mathbf{T}, \mathbf{B}, \mathbf{N}$, and curvature $\kappa$}
	
	\begin{itemize}
		\item $\mathbf{T}$: unit tangent vector
		$$\mathbf{T} = \dfrac{d\mathbf{r}}{ds} = \dfrac{\mathbf{v}}{|\mathbf{v}|} = {\color{red}\dfrac{d\mathbf{r}/dt}{|d\mathbf{r}/dt|}}$$
		
		\item $\kappa$: curvature
		$$\kappa = \left\vert \dfrac{d\mathbf{T}}{ds} \right\vert = \dfrac{|\mathbf{T}^\prime(t)|}{|\mathbf{r}^\prime(t)|} = \dfrac{|\mathbf{r}^\prime(t)\times\mathbf{r}^{\prime\prime}(t)|}{|\mathbf{r}^\prime(t)|^3} = \dfrac{|f^{\prime\prime}(x)|}{(1 + [f^\prime(x)]^2)^{\frac{3}{2}}} = \dfrac{|x^{\prime\prime}y^\prime - y^{\prime\prime}x^\prime|}{((x^\prime)^2 + (y^\prime)^2)^\frac{3}{2}}$$
		If $\kappa$ is large, $\mathbf{T}$ turns sharply, otherwise $\mathbf{T}$ turns slowly.
		
		\item $\mathbf{N}$: principal unit normal vector
		$$\mathbf{N} = \dfrac{1}{\kappa}\dfrac{d\mathbf{T}}{ds} = {\color{red}\dfrac{d\mathbf{T}/dt}{|d\mathbf{T}/dt|}}$$
		
			\item $\mathbf{B}$: binormal vector
			$$\mathbf{B} = \mathbf{T}\times \mathbf{N}$$
			
			\item Three planes:
			\begin{figure}[H]
				\centering
			\includegraphics[scale = 1.9]{myplane}
			\end{figure}
			\noindent 
			The curvature can be thought of as the rate at which
			the normal plane turns as P moves along its path.
	\end{itemize}
		
\end{frame}
















\section{Lecture 7: Planetary Motion in Polar coordinates}


\begin{frame}[allowframebreaks]{Kepler's Law}
	Kepler's Law
	\begin{enumerate}
		\item First law ({\color{blue} Law of Orbits}).
		$$$${\it \large The orbit of a planet is an ellipse with the Sun at one of the two foci.}
		$$$$E.g. Satellite motion around the Earth:
		\begin{figure}[H]
			\centering
			\includegraphics[scale = 0.30]{ec}
		\end{figure}
		\item Second law ({\color{blue}Law of Areas}). 
		$$$${\it \large A line segment joining a planet and the Sun sweeps out equal areas
		during equal intervals of time.}
	
	\begin{center}
		\animategraphics[scale = 0.5,autoplay,loop]{5}{name}{1}{48}
		
	\end{center}
	
	
		\item Third law ({\color{blue}Law of Periods}).
		$$$${\it \large The square of the orbital period of a planet is proportional to the
		cube of the semi-major axis of its orbit.}
		$$T^2 = \dfrac{4\pi^2}{GM}a^3~~~\Rightarrow ~~~T = \dfrac{2\pi}{\sqrt{GM}}a^{\frac{3}{2}}$$
	\end{enumerate}
	
\end{frame}


\end{document}