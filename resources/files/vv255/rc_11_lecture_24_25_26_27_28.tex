\documentclass[10pt]{beamer}
\usetheme{default}
\usepackage{float}
\usepackage{animate}
\usepackage{amsmath}

\usepackage{esvect}
%\usepackage[T1]{fontenc}

%IndentfFirst 
%\usepackage{indentfirst}
%\setlength{\parindent}{2em}
%\setlength{\parskip}{2em}


%Geometry
\usepackage{geometry}
\geometry{left = 0.25in,right = 0.25in}



%Adjust frametitle position (height,lateral position, etc)
\defbeamertemplate*{frametitle}{smoothbars theme}
{%
	%\nointerlineskip%
	\begin{beamercolorbox}[wd=\paperwidth,leftskip=.5cm,rightskip=.3cm plus1fil,vmode]{frametitle}
		\vskip +4.5ex
		\usebeamerfont*{frametitle}\insertframetitle%
		\vskip -0.9ex
	\end{beamercolorbox}%
}


\begin{document}


%headline

\setbeamertemplate{headline}{
\parbox{\linewidth}{\vspace*{8pt}\centering{ 
		    {\color{blue!40!black}\insertsection}
		}
	}
}


%footline
\setbeamertemplate{footline}[text line]{%
	\color{blue!40!black}\parbox{\linewidth}{\vspace*{-8pt}Michael Liu ~ (\insertshortinstitute)\hfill\insertshorttitle\hfill\insertshortdate~~~~~~\insertframenumber{}~/~\inserttotalframenumber}}

%Remove navigation symbols
\setbeamertemplate{navigation symbols}{}

\setbeamertemplate{frametitle continuation}[from second] 

\newcommand{\tabincell}[2]{\begin{tabular}{@{}#1@{}}#2\end{tabular}}

%\let\oldframe\frame\renewcommand\frame[1][allowframebreaks]{\oldframe[#1]}

%Title page	
\title[Vv255 Applied Calculus III]{Vv255 Applied Calculus III\\{\small Recitation XI (Final RC)}}   
\author[Michael Liu]{LIU Xieyang\\{\tiny Teaching Assistant}} 
\institute[UM-SJTU JI]{University of Michigan - Shanghai Jiaotong University \\Joint Institute}
\date[Summer 2015]{Summer Term 2015} 
\begin{frame}
	\titlepage
\end{frame}

%Table of Contents (All)
%\begin{frame}
%	\frametitle{Table of Contents}
%	\tableofcontents
%\end{frame}

%Table of contents (before, highlight each section)
\AtBeginSection[]{
	\begin{frame}
		\frametitle{Table of Contents}
		\frametitle{Contents}
		\tableofcontents[currentsection]
	\end{frame}}




%section
\section{Lecture 24: Divergence and Curl} 


\begin{frame}[allowframebreaks]{Divergence}
Suppose $\mathbf{F}(x, y, z)$ is a differentiable vector-valued function, where $x, y, z$ are
Cartesian coordinates, and let $P, Q, R$ be the components of $\mathbf{F}$. Then
$$\text{div}~\mathbf{F} = \nabla\cdot\mathbf{F} = \dfrac{\partial P}{\partial x} + \dfrac{\partial Q}{\partial y} + \dfrac{\partial R}{\partial z}$$
is called the {\color{blue}divergence} of $\mathbf{F}$ or the {\color{blue}divergence of the vector field} defined by $\mathbf{F}$.
\begin{itemize}
	\item $\nabla$ is pretend to be a vector in the dot product.
	\item A vector field which has {\color{red}zero} divergence is also referred to as {\color{blue}divergenceless or solenoidal}.
	\item Equation of continuity:
	$$\nabla\cdot \mathbf{u} + \dfrac{\partial \rho}{\partial t} = 0~~~~~(\text{Condition for conservation of mass})$$
	$$\nabla\cdot \mathbf{J} + \dfrac{\partial \rho}{\partial t} = 0~~~~~(\text{Condition for conservation of electric charge})$$
	\item The divergence measures outflow minus inflow.
\end{itemize}
To better motivate the meaning of ``divergence'', I provide with you a general definition:\\
{\bf We define the divergence of a vector field $\mathbf{F}$ at a point, $\nabla\cdot\mathbf{F}$, as the net outward flux of $\mathbf{F}$ per unit volume as the volume about the point tends to zero}:
$$\text{div}~\mathbf{F}\equiv\lim\limits_{\Delta v\rightarrow 0}\dfrac{\oint_S\mathbf{F}\cdot d\mathbf{S}}{\Delta v}$$
The numerator in the above equation, represents the net outward flux, is an integral over the entire surface $S$ that bounds the volume.
\end{frame}


\begin{frame}[allowframebreaks]{Curl}
Suppose $\mathbf{F}(x, y, z)$ is a differentiable vector-valued function, where $x, y, z$ are
Cartesian coordinates, and let $P, Q, R$ be the components of $\mathbf{F}$. Then
\[\text{curl}~\mathbf{F} = \nabla\times\mathbf{F} = \left\vert
\begin{matrix}
\mathbf{a}_x & \mathbf{a}_y & \mathbf{a}_z\\
\dfrac{\partial }{\partial x} & \dfrac{\partial }{\partial y} & \dfrac{\partial }{\partial z}\\
A_x & A_y & A_z
\end{matrix}
\right\vert\]
is called the {\color{blue}curl} of $\mathbf{F}$ or the {\color{blue}curl of the vector field} defined by $\mathbf{F}$.
\begin{itemize}
	\item $\nabla$ is pretend to be a vector in the cross product.
	\item Note that the curl of a vector field is a vector while the divergence is a scalar.
	\item The curl of the velocity field of a rotating rigid body has the direction of the axis of the rotation, and its magnitude equals {\color{red}twice} the angular speed of the rotation.
	\item Green's Theorem: $$\includegraphics[scale = 0.4]{green}$$
\end{itemize}
\end{frame}



\begin{frame}[allowframebreaks]{Null Identity}
Two identities involving repeated del operations are of considerable importance:
\begin{enumerate}
	\item IDENTITY 1:$$\nabla\times(\nabla V)\equiv 0$$
	where $V$ is a scalar function of two variables.\\
	In other words,
	 {\color{red} the curl of the gradient of any scalar filed is identically zero.} (The existence of $V$ and its first derivatives everywhere is implied here)\\
	 Then, we would have the following two expressions being valid:
	 \begin{enumerate}
	 	\item {\color{blue} If a vector field is curl-free, then it can be expressed as the gradient of a scalar field.}
	 	\item {\color{blue} An irrotational (a conservative, simply connected region assumed) vector field can always be expressed as the gradient of a scalar field.}
	 \end{enumerate}
	 Proof:\\
	 In general, if we take the surface integral of $\nabla\times(\nabla V)$ over any surface, the result is equal to the line integral of $\nabla V$ around the closed path bounding the surface (Stokes' Theorem):
	 $$\int_S[\nabla\times(\nabla V)]\cdot d\mathbf{s} = \oint_C(\nabla V)\cdot dl = 0$$
	 Note that the {\color{red}fundamental theorem of line integral} is invoked in the last step.
	 \item IDENTITY 2:$$\nabla\cdot(\nabla\times \mathbf{F}) \equiv 0$$
	 where $V$ is a scalar function of two variables.\\
	 In other words,
	 	 {\color{red} the divergence of the curl of any vector filed is identically zero.}\\
	 	  Then, we would have the following expression being valid:
	 	  \begin{enumerate}
	 	  	\item {\color{blue}If a vector field is divergenceless, then it can be expressed as the curl of another vector field.}
	 	  \end{enumerate}
\end{enumerate}

\end{frame}




\begin{frame}[allowframebreaks]{div \& curl in Cylindrical and Spherical Coordinates}

$$\includegraphics[scale = 0.4]{cy}$$
$$\includegraphics[scale = 0.35]{sp}$$	
$$\text{or}$$
\newpage
Cylindrical:
$$\nabla\cdot\mathbf{F} = \dfrac{1}{r}\dfrac{\partial}{\partial r}\left(rF_r\right) + \dfrac{1}{r}\dfrac{\partial F_\theta}{\partial \theta} + \dfrac{\partial F_z}{\partial z}$$
\[ \nabla\times\mathbf{F} = \dfrac{1}{r}\left\vert
\begin{matrix}
\mathbf{a}_r & \mathbf{a}_\theta r & \mathbf{a}_z\\
\dfrac{\partial }{\partial r} & \dfrac{\partial }{\partial \theta} & \dfrac{\partial }{\partial z}\\
A_r & rA_\theta & A_z
\end{matrix}
\right\vert\]
Spherical:
$$\nabla\cdot\mathbf{F} = \dfrac{1}{\rho^2}\dfrac{\partial}{\partial \rho}\left(\rho^2F_\rho\right) + \dfrac{1}{\rho\sin\phi}\dfrac{\partial}{\partial \phi}\left(F_\phi\sin\phi\right) + \dfrac{1}{\rho\sin\phi}\dfrac{\partial F_\theta}{\partial \theta}$$
\[ \nabla\times\mathbf{F} = \dfrac{1}{\rho^2\sin\phi}\left\vert
\begin{matrix}
\mathbf{a}_\rho & \mathbf{a}_\phi \rho & \mathbf{a}_\theta\rho\sin\phi\\
\dfrac{\partial }{\partial \rho} & \dfrac{\partial }{\partial \phi} & \dfrac{\partial }{\partial \theta}\\
A_\rho & \rho A_\phi & \rho\sin\phi A_\theta
\end{matrix}
\right\vert\]
\end{frame}



\section{Lecture 25: Parametric Surfaces and their areas} 
\begin{frame}[allowframebreaks]{Parametric Surfaces}
In much the same way that we describe a space curve by a vector function $\mathbf{r}(t)$ of a single
parameter $t$, we can describe a surface by a vector function $\mathbf{r}(u,v)$ of two parameters $u$ and $v$. We suppose that
$$\mathbf{r}(u,v) = x(u,v)\mathbf{a}_x +y(u,v)\mathbf{a}_y + z(u,v)\mathbf{a}_z$$
is a vector-valued function defined on a region $D$ in the $uv$-plane. So, $x, y,$ and $z$, the component functions of $\mathbf{r}$, are functions of the two variables $u$ and $v$ with domain $D$. The set of all points $(x, y, z)$ in $\mathbb{R}^3$ such that 
$$x = x(u,v)~~~~~y = y(u,v)~~~~~z = z(u,v)$$
and $(u,v)$ varies throughout $D$, is called a {\color{red} parametric surfaces} $S$, and the above equation of $\mathbf{r}$ is called {\color{red}parametric equations}. Each choice of $u$ and $v$ will give a point on $S$; by making all the choices, we get all of $S$. In other words, the surfaces $S$ is traced out by the tip of the position vector $\mathbf{r}(u,v)$ as $(u,v)$ moves throughout the region $D$.
$$\includegraphics[scale = 0.3]{parasurf}$$
For example, 
\[\begin{aligned}
\mathbf{r}(u,v) &= 2\cos u\mathbf{a}_x + v\mathbf{a}_y + 2\sin u\mathbf{a}_z \\
0&\leq u\leq \pi/2~~0\leq v\leq 3
\end{aligned}~~~\Rightarrow~~~
\begin{aligned}
\includegraphics[scale = 0.3]{cylinder}
\end{aligned}\]

\end{frame}


\begin{frame}[allowframebreaks]{Smoothness}
A parametric surface 
$$\mathbf{r}(u,v) = x(u,v)\mathbf{a}_x +y(u,v)\mathbf{a}_y + z(u,v)\mathbf{a}_z$$
is {\color{red}smooth} if 
\begin{enumerate}
	\item $\mathbf{r}_u$ and $\mathbf{r}_v$ are continuous
	\item $(\mathbf{r}_u\times\mathbf{r}_v)\neq\mathbf{0}$ on the interior of the parameter domain
	\begin{enumerate}
		\item the two vectors $\mathbf{r}_u$ and $\mathbf{r}_v$ are non-zero
		\item the two vectors $\mathbf{r}_u$ and $\mathbf{r}_v$ never lie along the same line
		\item There is alway a plane tangent to the surface. The normal vector of the tangent plane is given by $\mathbf{n} = \mathbf{r}_u\times\mathbf{r}_v$ which means that the equation of the tangent plane is $$n_1(x-x_0) + n_2(y-y_0) + n_3(z-z_0) = 0$$
	\end{enumerate}
\end{enumerate}
\end{frame}

\begin{frame}[allowframebreaks]{Surface Area}
The $\mathbf{r}_u$ and $\mathbf{r}_v$ are helpful for computing the surface area $S$.\\
For simplicity, we assume that $S$ is parametrized by
$$\mathbf{r}(u,v)~~~~\text{with domain} ~R = [a,b]\times[c,d]~\text{is a rectangle in the}~uv-\text{plane}$$
$$\includegraphics[scale = 0.3]{surfacearea1}$$
And the area of $S_{ij}$ could be approximated by the area of the parallelogram
$$\includegraphics[scale = 0.3]{surfacearea2}$$
$$|(\Delta u\mathbf{r}_u^*)\times (\Delta v\mathbf{r}_v^*)| = |\mathbf{r}_u^*\times \mathbf{r}_v^*|\Delta u\Delta v$$
and so an approximation to the area of $S$ is 
$$\sum\limits_{i=1}^m\sum\limits_{j=1}^n|\mathbf{r}_u^*\times \mathbf{r}_v^*|\Delta u\Delta v$$
By transforming it into a Riemann integral:
$$A(S) = \iint_R|\mathbf{r}_u\times \mathbf{r}_v|dA$$
\end{frame}

\begin{frame}[allowframebreaks]{Surface Area of the graph of a function}
If a surface is defined explicitly by $z = f (x, y)$, then we can parametrize it using
$$x = u, ~~~y = v, ~~~z = f(u,v)$$
and $$|\mathbf{r}_u\times\mathbf{r}_v| = \sqrt{f_u^2+f_v^2+1}$$
which is exactly $$\sqrt{f_x^2+f_y^2+1}$$
and it follows that $$S = \iint_R\sqrt{f_x^2+f_y^2+1} dA$$
which is exactly the same as what we've derived in RC\_07.
\end{frame}

\begin{frame}[allowframebreaks]{Surface of Revolution}

A surface of revolution S that is obtained by revolving the curve
$$y = f(x), ~~~~~a\leq x\leq b$$
around the $x$-axis has parametric equations $$x = u~~~y=f(u)\cos v~~~z = f(u)\sin v$$
where $a\leq u\leq b$ and $0\leq v \leq 2\pi$. And $$|\mathbf{r}_u\times\mathbf{r}_v| = |f(u)|\sqrt{1 + (f^\prime(u))^2}$$
and $$S = 2\pi\int_a^b|f(x)|\sqrt{1 + (f^\prime(x))^2}dx$$
using the fact that $x = u$.
$$\includegraphics[scale = 0.2]{rev}$$

\end{frame}


\section{Lecture 26: Surfaces Integrals}

\begin{frame}[allowframebreaks]{Surface Integral of Scalar Field}
The relationship between surface integrals and surface area is much the same as the relationship
between line integrals and arc length. Suppose $f$ is a function of three variables
whose domain includes a surface $S$. We divide $S$ into patches $S_{ij}$ with area $\Delta S_{ij}$. We evaluate $f$ at point $P_{ij}^*$ in each patch, multiply by the area $\Delta S_{ij}$, and form the sum
$$\sum\limits_{i=1}^m\sum\limits_{j=1}^nf(P_{ij}^*)\Delta S_{ij}$$
Then we take the limit as the patch size approaches $0$ and define the surface integral of $f$
over the surface $S$ as
$$\iint_Sf(x,y,z)dS = \lim\limits_{m,n\rightarrow\infty}\sum\limits_{i=1}^m\sum\limits_{j=1}^nf(P_{ij}^*)\Delta S_{ij}$$$$$$
Now, suppose that a surface $S$ has a vector equation
$$\mathbf{r}(u,v) = x(u,v)\mathbf{a}_x + y(u,v)\mathbf{a}_y + z(u,v)\mathbf{a}_z~~~~~(u,v)\in D$$
$$\includegraphics[scale = 0.8]{surfaceintegral2}~~~~\includegraphics[scale = 0.8]{surfaceintegral1}$$
Note that we already know from last section that $\Delta S_{ij}\rightarrow|\mathbf{r}_u\times\mathbf{r}_v|dudv$
Thus, 
$$\iint_Sf(x,y,z)dS = \iint_Df(\mathbf{r}(u,v))|\mathbf{r}_u\times\mathbf{r}_v|dA$$
{\color{purple}
It actually makes sense comparing with line integral: $$\int_Cf(x,y,z)ds = \int_a^bf(\mathbf{r}(t))|\mathbf{r}^\prime(t)|dt$$}
If $S$ is a piecewise-smooth surface, that is, a finite union of smooth surfaces $S_1$, $S_2$, $\cdots$, $S_n$ that intersect only along their boundaries, then the surface integral of $f$ over $S$ is defined
by
$$\iint_Sf(x,y,z)dS = \iint_{S_1}f(x,y,z)dS + \iint_{S_2}f(x,y,z)dS + \cdots + \iint_{S_n}f(x,y,z)dS$$
\end{frame}



\begin{frame}[allowframebreaks]{Oriented Surface}
For an orientable (two-sided) surface $S$ that has a tangent plane everywhere except at any boundary points), there are two unit normal vectors $\mathbf{n}_1$ and $\mathbf{n}_2$ at an arbitrary point $(x,y,z)$.
$$\includegraphics[scale = 0.2]{oriented}$$
If it is possible to choose a unit normal vector $\mathbf{n}$ at every such point $(x,y,z)$ so that $\mathbf{n}$ varies continuously over $S$, then $S$ is called an {\bf oriented surface} and the given choice of $\mathbf{n}$ provides $S$ with an {\bf orientation}.
\\For a smooth orientable surface given in {\color{red}parametric form} by a vector function $\mathbf{r}(u,v)$, then it is automatically supplied with the orientation of the unit normal vector
$$\mathbf{n} = \dfrac{\mathbf{r}_u \times\mathbf{r}_v}{|\mathbf{r}_u\times\mathbf{r}_v|}$$
as can be inherited from the previous section about surface areas.
	
\end{frame}


\begin{frame}[allowframebreaks]{Surface Integral of Vector Field}
For a general vector field $\mathbf{F}$, we define the surface integral  of $\mathbf{F}$ over $S$ by $$\iint_S\mathbf{F}\cdot d\mathbf{S} = \iint_S\mathbf{F}\cdot \mathbf{n}~dS$$
This integral is also called the {\bf flux} of $\mathbf{F}$ across $S$.
We are sort of only care about the normal component of $\mathbf{F}$ (component of $\mathbf{F}$ along the direction of the normal vector $\mathbf{n}$) at a point, which is illustrated by
$$\includegraphics[scale = 0.25]{normalcomponent}$$
Recall that $$\mathbf{n} = \dfrac{\mathbf{r}_u \times\mathbf{r}_v}{|\mathbf{r}_u\times\mathbf{r}_v|}$$
in a parametric surface, we could have another formula to calculate the surface integral of a vector field:
$$\includegraphics[scale = 0.4]{normalformula}$$
\newpage
Applications:
\begin{enumerate}
	\item Gauss's Law:
	$$\oint_\mathbf{S}\mathbf{E}\cdot d\mathbf{S} = \dfrac{Q}{\epsilon_0}$$
	which says that {\bf the total outward flux of the electric field intensity over any closed surface in free space is equal to the total charge enclosed in the surface divided by $\epsilon_0$}.
	\\{\small\color{red} Note that $\oint_{\mathbf{S}}$ denote a closed surface integral over a vector field}.
	\item Calculating outward flux of a vector field with a {\color{red}cylindrical and spherical symmetry}. For example: The total outward flux of $\mathbf{F} = 5\mathbf{a}_\rho$ over a spherical surface of radius 3 centered at origin is:
	$$\includegraphics[scale = 0.2]{flux}$$
	\begin{align*}
	\oint_S\mathbf{F}\cdot d\mathbf{S} &= \oint_S5\mathbf{a}_\rho\cdot \mathbf{a}_\rho dS
	\\&=4\pi\cdot 3^2\cdot 5
	\\&=180\pi
	\end{align*}
\end{enumerate}
	
\end{frame}



\section{Lecture 27: Stokes' Theorem}

\begin{frame}[allowframebreaks]{Stokes' Theorem}
$$\includegraphics[scale = 0.4]{stoke}$$
-Stokes' Theorem can be regarded as a higher-dimensional version of Green's Theorem.\\
-For any infinitesimal patch, we could apply the tangential form of the Green's Theorem
$$\oint_C\mathbf{F}\cdot \mathbf{T}ds = \iint\limits_D(\nabla\times\mathbf{F})dA$$
Then, we summing the above integral up:\\
Note that the common part of the contours of two adjacent patches is traversed in opposite directions by two contours, the net contribution of all the common parts in the interior to the total line integral is zero, only the contribution from the external contour $C$ bounding the entire area $S$ remains after the summation. And the Stokes' Theorem states:\\~\\
Let $S$ be an oriented piecewise smooth surface that is bounded by a positively
oriented, piecewise smooth, simple, closed boundary curve $C$. Let $\mathbf{F}$ be a vector field whose components have continuous partial derivatives on an open region in
$\mathbb{R}^3$ that contains $S$. Then
$$\iint_S(\nabla\times\mathbf{F})\cdot d\mathbf{S} = \iint_S(\nabla\times\mathbf{F})\cdot\mathbf{n}dS = \oint_C\mathbf{F}\cdot d\mathbf{r}$$
where $\mathbf{n}$ is the unit normal vector of $S$.\\
It states that {\color{red} the surface integral of the curl of a vector field over an open surface is equal to the closed line integral of the vector along the contour bounding the surface}.
$$\includegraphics[scale = 0.1]{closed}$$
Particularly, if the surface integral of $\nabla\times\mathbf{F}$ is carried over a closed surface, there will be no surface-bounding external contour, and the above equation tells that $$\oint_S(\nabla\times\mathbf{F})\cdot d\mathbf{S} = 0$$
by the fundamental theorem of line integral.


\end{frame}






\section{Lecture 28: Divergence Theorem}
\begin{frame}[allowframebreaks]{Divergence Theorem}
In previous sections, we found that the {\color{red}normal form of Green's Theorem} in a vector version is 
$$\int_C\mathbf{F}\cdot\mathbf{n}ds = \iint\limits_D(\nabla\cdot\mathbf{F})dA$$
where $C$ is the positively oriented boundary curve of the plane region $D$. If we seek to extend this theorem to vector fields in $\mathbb{R}^3$, we might make the guess that 
$$\iint\limits_S\mathbf{F}\cdot\mathbf{n}dS = \iiint\limits_E(\nabla\cdot\mathbf{F})dV$$
where $S$ is the boundary surface of the solid region $E$. It turns out that the above equation is true, under appropriate hypotheses, and is called the $Divergence ~ Theorem$. {\color{blue}Note that its similarity
	to Green's Theorem and Stokes' Theorem in that it relates the integral of a derivative of a
	function ( $\nabla\cdot\mathbf{F}$ in this case) over a region to the integral of the original function $\mathbf{F}$ over the
	boundary of the region}.
$$\includegraphics[scale = 0.3]{div}$$
To motivate that definition, we consider an infinitesimal  volume $\Delta v_j$ bounded by a surface $s_j$, by applying the definition of divergence
$$(\nabla\cdot\mathbf{F})\Delta v_{j} = \oint_{s_j}\mathbf{F}\cdot d\mathbf{S}$$$$$$$$$$$$$$
Now, sum the above integrals up (by taking the volume integral of both sides):\\
The contributions from the internal surfaces of adjacent infinitesimal volumes will however, cancel each other, because at a common internal surface the outward normals of the adjacent volumes point in opposite directions. Hence the net contributions of the sum of the right side is due only to that of the external surface $S$ bounding the volume $V$, that is:
$$\iiint_E(\nabla\cdot\mathbf{F})dV = \oint_S\mathbf{F}\cdot d\mathbf{S} $$$$$$$$$$$$$$
Now, we formally states that {\color{blue}Divergence Theorem} as:\\~
\\Let $\mathbf{F}$ be a vector field having continuously differentiable components, and let $S$
be a piecewise smooth oriented closed surface enclosing a solid region $E$, then
$$\iiint_E(\nabla\cdot\mathbf{F})dV = \iint_S\mathbf{F}\cdot d\mathbf{S} = \oint_S\mathbf{F}\cdot\mathbf
{n}dS$$
where $\mathbf{n}$ is the unit normal vector of $S$.\\
It states that {\color{red}the volume integral of the divergence of vector field equals the total outward flux of the vector through the surface that bounds the volume}.
$$$$
For the exercises of Divergence and Stokes' Theorem, please refer to the textbook.

\end{frame}



\section{Goodbye!}
\begin{frame}[allowframebreaks]{Goodbye!}
\begin{figure}[H]
	\centering
	\includegraphics[scale = 0.4]{qian}~
	\includegraphics[scale = 0.4]{yu}~
	\includegraphics[scale = 0.4]{mao}\\~\\
	\includegraphics[scale = 0.4]{hou}
\end{figure}
	
\end{frame}


\end{document}