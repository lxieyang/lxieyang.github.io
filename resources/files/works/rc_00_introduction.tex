\documentclass{beamer}
\usetheme{default}
\usepackage{float}
\usepackage{animate}
%\usepackage[T1]{fontenc}

%IndentfFirst 
%\usepackage{indentfirst}
%\setlength{\parindent}{2em}
%\setlength{\parskip}{2em}


%Geometry
\usepackage{geometry}
\geometry{left = 0.25in,right = 0.25in}



%Adjust frametitle position (height,lateral position, etc)
\defbeamertemplate*{frametitle}{smoothbars theme}
{%
	%\nointerlineskip%
	\begin{beamercolorbox}[wd=\paperwidth,leftskip=.5cm,rightskip=.3cm plus1fil,vmode]{frametitle}
		\vskip +4.5ex
		\usebeamerfont*{frametitle}\insertframetitle%
		\vskip -0.9ex
	\end{beamercolorbox}%
}


\begin{document}


%headline

\setbeamertemplate{headline}{
\parbox{\linewidth}{\vspace*{8pt}\centering{ 
		    {\color{blue!40!black}\insertsection}
		}
	}
}


%footline
\setbeamertemplate{footline}[text line]{%
	\color{blue!40!black}\parbox{\linewidth}{\vspace*{-8pt}Michael Liu ~ (\insertshortinstitute)~~~~~~~~~~~~~~~~~~~~~~~~\insertshorttitle\hfill\insertshortdate~~~~~~\insertframenumber{}~/~\inserttotalframenumber}}

%Remove navigation symbols
\setbeamertemplate{navigation symbols}{}

\setbeamertemplate{frametitle continuation}[from second] 

\newcommand{\tabincell}[2]{\begin{tabular}{@{}#1@{}}#2\end{tabular}}

%\let\oldframe\frame\renewcommand\frame[1][allowframebreaks]{\oldframe[#1]}

%Title page	
\title[Vv255 Applied Calculus III]{Vv255 Applied Calculus III\\{\small Introduction}}   
\author[Michael Liu]{LIU Xieyang\\{\tiny Teaching Assistant}} 
\institute[UM-SJTU JI]{University of Michigan - Shanghai Jiaotong University \\Joint Institute}
\date[Summer 2015]{Summer Term 2015} 
\begin{frame}
	\titlepage
\end{frame}

%Table of Contents (All)
\begin{frame}
	\frametitle{Table of Contents}
	\tableofcontents
\end{frame}

%Table of contents (before, highlight each section)
%\AtBeginSection[]{
%	\begin{frame}
		%\frametitle{Table of Contents}
%		\frametitle{Contents}
%		\tableofcontents[currentsection]
%	\end{frame}




%section
\section{Course Introduction} 



\begin{frame}[allowframebreaks]{Course Introduction}
	Vv255 Applied Calculus III generally talks about the following topics in the field of multi-variable calculus:
\begin{itemize}
	\item Vectors, Matrices and Linear equations and their respective operations
	\item Parametric Equations \& Vector-valued functions
	\item Arc length and Curvature
	\item Directional derivatives and Gradient vector
	\item Double \& Triple integrals in different coordinate systems and their applications
	\item Vector fields, line integrals and surface integrals
	\item Curl and divergence
\end{itemize}

\end{frame}













\section{Office Hours, TAs, Emails}

\begin{frame}[allowframebreaks]{Office Hours, TAs, Emails}
\begin{itemize}
		\item Instructor: Prof. Jing Liu 
		\begin{itemize}
			\item Email: stephen.liu@sjtu.edu.cn
			\item Office Hour: Tuesday and Thursday (2pm -- 4:30pm) in Room 204 or by appointment.
		\end{itemize}
		
		\item Teaching Assistants:{\small
		\begin{table}[H]
			\begin{tabular}{|c||c|c|}\hline
				Name & Email &  \tabincell{c}{Recitation\\ Session} \\\hline\hline
				\tabincell{c}{HOU Yijun} & susan57@sjtu.edu.cn & Thur. 18:20 -- 20:00 \\\hline
				\tabincell{c}{LIU Xieyang} & michael.liu@sjtu.edu.cn & Fri. 12:10 -- 13:50 \\\hline
				\tabincell{c}{LU Yuchen} & lyc1102@sjtu.edu.cn & Mon. 18:20 -- 20:00 \\\hline
				\tabincell{c}{QIAN Junqi} & kyle1994@sjtu.edu.cn & Wed. 18:20 -- 20:00 \\\hline
			\end{tabular}
		\end{table}}
		\noindent 
		The office hour of each TA will be the last 40 minutes of each recitation session.
\end{itemize}
\end{frame}















\section{Coursework Policy}

\begin{frame}[allowframebreaks]{Coursework Policy}
The following applies mostly to the paper-based homework at the
beginning of this term:
\begin{itemize}
\item \alert{Hand in your coursework on time, by the date given on each set of course work. Late work will not be accepted unless valid reasons are provided prior to the due date of each set of course work.}
\item Please write your course work neatly and legibly. \alert{Your marks could be deducted if you failed to hand in a neat and readable assignment solution.}
\item You are \alert{encouraged} to provided your {\it Matlab} codes (Yes, {\it Matlab}. Do you miss it?) along with your computer-generated images or results of any form in the solution. Though not counted as bonus of any form, it will greatly help the TAs to better examine your {\it Matlab} skills.
\end{itemize}
\end{frame}


\section{Use of Wikipedia and Other Sources; Honor Code Policy}
\begin{frame}[allowframebreaks]{Use of Wikipedia and Other Sources; Honor Code Policy}
When faced with a particularly difficult problem, you may want to refer to
other textbooks or online sources such as Wikipedia. Here are a few
guidelines:
\begin{itemize}
\item Outside sources may treat a similar sounding subject matter at a
much more advanced or a much simpler level than this course. This
means that explanations you find are much more complicated or far
too simple to help you. For example, when looking up the ``divergence'' you may find many high-school level explanations that are not
sufficient for our problems; on the other hand, wikipedia contains a
lot of information relating to formal logic that is far beyond what we
are discussing here.
\item If you do use any outside sources to help you solve a homework
problem, you are not allowed to just copy the solution; this is
considered a violation of the Honor Code.
\item The correct way of using outside sources is to understand the
contents of your source and then to write in your own words and
without referring back to the source the solution of the problem. Your
solution should differ in style significantly from the published solution.
If you are not sure whether you are incorporating too much material
from your source in your solutions, then you must cite the source that
you used.
\item You may cooperate with other students in finding solutions to
assignments, but you must write your own answers. Do not simply
copy answers from other students. It is acceptable to discuss the
problems orally, but you may not look at each others' written notes.
Do not show your written solutions to any other student. This would
be considered a violation of the Honor Code.
\end{itemize}

In this course, the following actions are examples of violations of the
\alert{Honor Code}:
\begin{itemize}
\item Showing another student your written solution to a problem.
\item Sending a screenshot of your solution via QQ, email or other means
to another student.
\item  Showing another student the written solution of a third student;
distributing some student's solution to other students.
\item Viewing another student's written solution.
\item Copying your solution in electronic form (codes, PDF, JPG
image etc.) to the computer hardware (flash drive, hard disk etc.) of
another student. Having another student's solution in electronic form
on your computer hardware.
\end{itemize}
\end{frame}



\section{Course Grade}
\begin{frame}[allowframebreaks]{Course Grade}
	
	\begin{itemize}
		\item  Quiz (5\%)
		\item  Assignment (20\%)
		\item  Midterm Exam 1 (25\%)
		\item  Midterm Exam 2 (25\%)
		\item  Final Exam (25\%)
	\end{itemize}
	The grade will be curved to achieve a median grade of ``\alert{B}''.
	\\~\\~\\~
	\begin{center}
		Well, ...
	\end{center}
	\newpage
	\begin{figure}[H]
		\centering
		\includegraphics[scale = 0.3]{193.png}
	\end{figure}
	\noindent 
	\\~\\
\begin{center}
	
	Take care of yourselves.
\end{center}	
	
\end{frame}



\section{Surprise!}
\begin{frame}[allowframebreaks]{Surprise!}
	
	\begin{center}
		\animategraphics[scale = 0.3,autoplay,loop]{20}{name}{1}{100}
		\\~\\No pains, no gains. \\And stop day-dreaming, no surprises will fall out of the sky.
	\end{center}
	
	%这里的n表示我有n张的png文件,而且png文件的名字是从name1.png,name2.png,...开始到namen.png。
	%{15}设置播放速度
	%loop不用说了就是一直这几张png文件循环放映。
	
\end{frame}



\end{document}