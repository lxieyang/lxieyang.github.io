\documentclass[10pt]{beamer}
\usetheme{default}
\usepackage{float}
\usepackage{animate}
\usepackage{amsmath}

\usepackage{esvect}
%\usepackage[T1]{fontenc}

%IndentfFirst 
%\usepackage{indentfirst}
%\setlength{\parindent}{2em}
%\setlength{\parskip}{2em}


%Geometry
\usepackage{geometry}
\geometry{left = 0.25in,right = 0.25in}



%Adjust frametitle position (height,lateral position, etc)
\defbeamertemplate*{frametitle}{smoothbars theme}
{%
	%\nointerlineskip%
	\begin{beamercolorbox}[wd=\paperwidth,leftskip=.5cm,rightskip=.3cm plus1fil,vmode]{frametitle}
		\vskip +4.5ex
		\usebeamerfont*{frametitle}\insertframetitle%
		\vskip -0.9ex
	\end{beamercolorbox}%
}


\begin{document}


%headline

\setbeamertemplate{headline}{
\parbox{\linewidth}{\vspace*{8pt}\centering{ 
		    {\color{blue!40!black}\insertsection}
		}
	}
}


%footline
\setbeamertemplate{footline}[text line]{%
	\color{blue!40!black}\parbox{\linewidth}{\vspace*{-8pt}Michael Liu ~ (\insertshortinstitute)\hfill\insertshorttitle\hfill\insertshortdate~~~~~~\insertframenumber{}~/~\inserttotalframenumber}}

%Remove navigation symbols
\setbeamertemplate{navigation symbols}{}

\setbeamertemplate{frametitle continuation}[from second] 

\newcommand{\tabincell}[2]{\begin{tabular}{@{}#1@{}}#2\end{tabular}}

%\let\oldframe\frame\renewcommand\frame[1][allowframebreaks]{\oldframe[#1]}

%Title page	
\title[Vv255 Applied Calculus III]{Vv255 Applied Calculus III\\{\small Recitation VI}}   
\author[Michael Liu]{LIU Xieyang\\{\tiny Teaching Assistant}} 
\institute[UM-SJTU JI]{University of Michigan - Shanghai Jiaotong University \\Joint Institute}
\date[Summer 2015]{Summer Term 2015} 
\begin{frame}
	\titlepage
\end{frame}

%Table of Contents (All)
%\begin{frame}
%	\frametitle{Table of Contents}
%	\tableofcontents
%\end{frame}

%Table of contents (before, highlight each section)
\AtBeginSection[]{
	\begin{frame}
		\frametitle{Table of Contents}
		\frametitle{Contents}
		\tableofcontents[currentsection]
	\end{frame}}




%section
\section{Lecture 12: Differential} 



\begin{frame}[allowframebreaks]{Linearization \& Differential}
~\\
{\color{blue} Linearization}:\\
The {\color{red}linearization} of a function $f (x, y)$ at a point $(x_0, y_0)$, where $f$ is differentiable,
$$L(x,y) = f(x_0,y_0) + f_x(x_0,y_0)(x-x_0) + f_y(x_0,y_0)(y-y_0).$$
$$$$$$$${\color{blue}Differential}:
If we move from$ (x_0, y_0)$ to a point $(x_0 +dx, y_0 +dy)$ nearby, the resulting change
$$df = f_x(x_0,y_0)dx + f_y(x_0,y_0)dy$$
in the linearization of $f$ is called the  {\color{red}total differential} of $f$.

\newpage
\begin{itemize}
	\item The plane $z = L(x, y)$ is tangent to the surface $z = f (x, y)$ at the point $(x_0, y_0)$		.
	\item The linearization of a function of two variables is a {\color{red}tangent-plane approximation}
	in the same way that the linearization of a function of a single variable is a
	{\color{blue}tangent line approximation}.
\end{itemize}

\end{frame}



















\section{Lecture 13: Extreme values and saddle points}

\begin{frame}[allowframebreaks]{Extreme Values for function of several variables}
	\begin{figure}[H]
		\centering
		\includegraphics[scale = 0.4]{extremedef}
	\end{figure}

\end{frame}

\begin{frame}[allowframebreaks]{Finding Local Extrema}
Procedures for finding {\color{red}local/relative} extrema:\\
Step 1: 
{\color{blue}Gradient Test: Finding \color{red}Critical Points}
\begin{figure}[H]
	\centering
	\includegraphics[scale = 0.2]{criticalpoint}
\end{figure}

\begin{figure}[H]
	\centering
	\includegraphics[scale = 0.22]{saddle}
	\caption{{\color{red}Saddle point:} In general, we will say that a function has a saddle point P if there are two
		distinct vertical planes through P such that P in one of the planes is a
		local maximum and P in the other is a local minimum.}
	\end{figure}
	
Step 2: 
{\color{blue}Second derivative Test: {\color{red}Hessian Matrix}}
\begin{figure}[H]
	\centering
	\includegraphics[scale = 0.3]{hessian}
\end{figure}
\begin{figure}[H]
	\centering
	\includegraphics[scale = 0.35]{seconddtest}
\end{figure}

	
\end{frame}



\begin{frame}[allowframebreaks]{Finding Global Extrema}
Procedures for finding {\color{red}global/Absolute} extrema:
\begin{enumerate}
	\item Find the local extreme values of $f$ in the domain $D$.
	\item Find the local extreme values of $f$ on boundary of the domain $D$.
	\begin{itemize}
		\item Direct examination;
		\item Lagrange multiplier (will be talked about later)
	\end{itemize}
	\item Compare values in step 1. and step 2., the largest of them is the global
	maximum, the smallest is the global minimum.
\end{enumerate}

	
\end{frame}










\section{Lecture 14: Lagrange Multipliers}

\begin{frame}[allowframebreaks]{Geometry Basis of Lagrange Multipliers}~\\
\begin{figure}[H]
	\centering
	\includegraphics[scale = 0.25]{lm1}~~~
	\includegraphics[scale = 0.25]{lm2}
	\caption{Maximum of $f$ is 400; minimum of $f$ is 200}
\end{figure}
To motivate the method of Lagrange multipliers, suppose that we are trying to maximize
a function $f(x,y)$ subject to the constraint $g(x, y) = 0$. Geometrically, this means that we
are looking for a point $(x_0, y_0)$ on the graph of the constraint curve at which $f(x, y)$ is as
large as possible.\newpage
\begin{figure}[H]
	\centering
	\includegraphics[scale = 0.22]{lm1}~~~
	\includegraphics[scale = 0.22]{lm2}
	\caption{Maximum of $f$ is 400; minimum of $f$ is 200}
\end{figure}
To help locate such a point, let us construct a contour plot of $f(x, y)$ in the same coordinate system as the graph of $g(x, y) = 0$. For example, the graph on the left shows some typical level curves  of $f(x,y) = c$, which we have labeled $c = 100,200,300,400,500$ for purpose of illustration. In this figure, each point of intersection of $g(x, y) = 0$
with a level curve is a candidate for a solution, since these points lie on the constraint curve. Among the seven such intersections shown in the figure, the maximum value of $f(x, y)$
occurs at the intersection $(x_0, y_0)$ where $f(x, y)$ has a value of 400.\newpage
\begin{figure}[H]
	\centering
	\includegraphics[scale = 0.22]{lm1}~~~
	\includegraphics[scale = 0.22]{lm2}
	\caption{Maximum of $f$ is 400; minimum of $f$ is 200}
\end{figure}
 Note that at $(x_0, y_0)$
the constraint curve and the level curve just {\color{red}touch} and thus have a {\color{red}common tangent line} at this point.
\\Since $\nabla f(x_0, y_0)$ is {\color{red} normal} to the level curve $f(x, y) = 400$ at $(x_0, y_0)$, and
since $\nabla g(x0, y0)$ is {\color{red} normal} to the constraint curve $g(x, y) = 0$ at $(x_0, y_0)$, we conclude that
the vectors $\nabla f(x_0, y_0)$ and $\nabla g(x_0, y_0)$ must be {\color{red} parallel}. That is 
$$\nabla f(x_0, y_0) = \lambda \nabla g(x_0, y_0)~~~~~~~\text{for some scalar } ~ \lambda$$
\end{frame}








\begin{frame}[allowframebreaks]{Lagrange Multipliers}~\\
{\color{blue}Ultimate Goal}:
\begin{center}
	Minimize/Maximize $f$ subject to the constraint(s) $g_1= k_1$, $g_2 = k_2$, $\cdots$, $g_n = k_n$, \\where $f$ and $g_1$, $\cdots$, $g_n$ are functions of several variables.
\end{center}

{\color{blue}Method}:\\Step 1: Solve the simultaneous equations:
\begin{center}
	$\nabla f = {\color{red}\lambda} \nabla g_1 + {\color{red}\mu} \nabla g_2 + \cdots$\\$g_1 = k_1$\\$g_2 = k_2$\\$\cdots$
\end{center}
\noindent
Step 2: 
\\Evaluate $f$ at all the points $(x,y,z,\cdots)$ that result from step 1. The largest of
these values is the maximum value of $f$ ; the smallest is the minimum value
of $f$.
	
\end{frame}


\end{document}