%Use a4paper in the documentclass
\documentclass[a4paper,11pt]{article}

\usepackage{bm}
\usepackage{textcomp}
\usepackage{appendix}
\usepackage{nicefrac}
\usepackage{units}
\usepackage{tikz}
%\usepackage{times}
\usepackage{amsmath}
\usepackage{amssymb}
%\usepackage[framed,numbered]{mcode}
%\DeclareMathOperator*{\SIGMA}{SIGMA}
%for 下划线 and 波浪线
\usepackage{ulem}
\usepackage{graphicx}
%Indent the first paragraph of each section*.
\usepackage{url}
\usepackage[colorlinks, linkcolor=black, anchorcolor=black, citecolor=black,urlcolor=black]{hyperref}
\usepackage{float}
%\usepackage{indentfirst}
\usepackage{setspace}
%set geometry of the article
\usepackage{longtable}
\usepackage{lastpage}
\usepackage{geometry}
\geometry{left=3.0cm,right=3.0cm,top=3.0cm,bottom=2.5cm}

%\usepackage{package}


\usepackage{palatino}

%页眉页脚
\usepackage{fancyhdr}
\pagestyle{fancy}
\lhead{Team \# 33804}
\chead{}
\rhead{Page \thepage ~of \pageref{LastPage}}
\lfoot{}
\cfoot{}
\rfoot{}




%\renewcommand{\familydefault}{\sfdefault} 

\newcommand{\tabincell}[2]{\begin{tabular}{@{}#1@{}}#2\end{tabular}}


\begin{document}
\title{\LARGE {\bf How to Carry Out An Effective Search Operation} \\ \tiny ~ \\ \large Response to Problem B: Searching for A Lost Plane}
\author{Team \# 33804}
\date{February 9, 2015}
\maketitle
%Spacing of the article. Package:setspace Do not FORGET \end{spacing}
\begin{spacing}{1.0}

\begin{abstract}
We constructed 3 mathematical models to guide the search and rescue operation for planes feared to have crashed in the ocean. 

Model I deals with the descending process of the airplane. By analyzing the free-body diagram, we obtained a system of differential equations which characterizes the motion of the airplane during descending. By specifying the initial conditions, we could solve the system by \textit{MATLAB}, and obtain the flying trajectory and possible crashing position.

Model II studies the sinking process of the wreckage and the black boxes. We consider three different scenarios where the angle between the ocean current and the crashing direction varies. Thus, we obtain three sets of free-body diagrams and systems of differential equations. By applying the results of model I as the initial conditions, the systems could be solved respectively. Therefore, the positions of the main wreckage and black boxes are acquired.

Model III provides an algorithm which guides the search operation. To cope with different situations encountered in the operation, we have 5 different plans at our disposal. Each plan is designed to search for survivors, floating wreckage, or main wreckage underwater respectively. During the beginning of the operation, the algorithm lays much emphasis on searching and rescuing survivors. As time goes by, the algorithm shifts its focus onto the recovery of bodies, wreckage, and black boxes.

To verify the effectiveness of our plan, we compare our plan with a reference plan by simulation in a computer program. As was expected, our plan surpasses the reference one in the times of successes. Therefore, the models we construct are valid and promising.



\end{abstract}

%\tableofcontents

\newpage
\section{Introduction}\label{Introduction}
Nearly 6 years ago, we witnessed one of the most horrible  air accident in the history of mankind. Air France Flight 447, which was scheduled to arrive at Paris, crashed into the Atlantic Ocean, killing all 228 souls on board. It took the search teams almost 2 years to locate and recover the flight recorders (black boxes) of the aircraft.\cite{af447} 5 years later, in 2014, the Malaysia Airlines Flight 370 was reported missing after deviating from its original course. Although massive searches for the jet have been carried out by international efforts in different oceans for a almost a year, not a single piece of the Boeing 777-200ER was found ever.
\\There have been many similar disasters in the past, and unfortunately our commercial airplanes are not able to ensure 100\% safety. In case of emergencies, it is still an imperative option to build a generic mathematical model that could assist ``searchers'' in planning a more effective, powerful, and useful search for a lost plane which feared to have crashed in open water such as the Atlantic, Pacific, Indian, Southern, or Arctic Ocean. The model should recognize that there are many different types of planes for which we might be searching for and that there are many different types of search planes with different electronics or sensors.
\\We first build two models to determine the approximated location where the lost plane hits the ocean from its last known position and the possible location where the main wreckage/black box lies on the seabed. Next we construct another model to optimize the search operation. Then, we apply our models to a specific air accident case, thus verifying the validity and feasibility of our approach. Also, our models are easily adaptable to planes of different features parameters like total mass and lifting coefficient.
\subsection{Defining the Problem}\label{Def the Problem}
\begin{itemize}
	\item How to determine the approximated  location where the ``possibly crashed'' aircraft hit the surface of the ocean from the position where it lost communication with air traffic control and encountered a irremediable technical failure?
	\item How to simulate the approximated movement of the wreckage and the black boxes once they are under water, thus anticipate the final position where they rest on the seabed? 
	\item What particular scheme should be adopted by the authorities and airliners to optimize search operations once the approximated location of the survivors and wreckage are determined?
\end{itemize}


\subsection{Model Overview}\label{Model Overview}
In order to tackle the problems above, we build the following models:
\begin{itemize}
	\item A model that receives  specific parameters and position information of an airplane and then predicts the possible location where it hits the surface of the ocean, known as the \textbf{position of contact} (\textit{POC}).
	\item A model that receives  specific parameters of the plane and information about the \textit{POC} and then determines the approximated location of the wreckage and black boxes on the seabed.
	\item A model that utilizes the output above to generate an optimized search plan with limited resources of search planes, helicopters, ships, and submarines. 
\end{itemize}


\subsection{Constraints}\label{Constraints}
The problem should be solved with the following constrains:
\begin{itemize}
	\item The lost planes could only crash into open water but not onto any form of land.
	\item There are absolutely no electronic signals like \textit{GPS} transmitted from the plane that could lead to the direct pinpoint of its position after the crash.

\end{itemize}



\subsection{Assumptions}\label{Assumptions}
During the process, we would like to operate under these proper assumptions:
\begin{itemize}
	\item The drag force always acts horizontally and opposite to the plane's flying direction.
	\item The plane loses all communications after the irremediable technical failure occurred.
	\item After the technical failure occurred, the plane loses all sorts of power (hence propulsion) and can no longer control its airspeed.
	\item During the rapid descending of altitude (in the air), the pilot somehow manages to keep the elevation angle of the plane stabilized.
	\item The surface of the ocean could be treated as a plane instead of a sphere.
	\item The descending process (in the air) takes place in the 2D-plane where the original flying direction lies.
	\item The acceleration due to gravity remains constant during the descending/sinking.
	\item The speed and direction of ocean current remains constant in one specific ocean, the speed does not change from the surface of the ocean to the bottom. The direction of ocean current always lies in the horizontal plane.
	\item The density of the ocean water $\rho_\text{oce}$ does not change with depth.
	\item The time from \textit{POC} to the wreckage and black boxes being just completely submerged underwater could be neglected.
	\item The wreckage/black box immediately loses all horizontal and vertical speed once hits the seabed. It simply rests on the seabed without being affected by ocean current, marine creatures, or other forms of underwater activities.
	\item The resources (planes, helicopters, ships, submarines) for searching are limited.
	\item The time for the search planes to maneuver to its designated searching area (either from an airport or its last searching area) could be neglected, since it is relatively little compared with the time consumed on searching.
	\item There are multiple search planes available. Search planes can function differently by reinstalling different equipment if necessary.
	\item Fixed wing airplanes (will be mentioned by ``plane'' later in the text) are mainly responsible for searching. Helicopters, ships and submarines are mainly responsible for rescue and salvage.
\end{itemize}



\section{Methods}\label{Methods}



%MODEL I: DESCENDING MODEL
\subsection{Model I: Descending Model}
The plane has lost all communication and power and will immediately begin its inevitable descending with an initial speed in the direction of its current motion. This model deals with the situation where the plane is descending without power in the air.
\subsubsection{Notations}
\begin{center}
	\begin{longtable}{cl}
		\caption{Notations (Model I)}\\
		\hline
		Symbol &  Meaning \\
		\hline\hline
		\endfirsthead
		\multicolumn{2}{c}%
		{\tablename\ \thetable\ -- \textit{Continued from previous page}} \\
		\hline
		Symbol &    Meaning \\
		\hline\hline
		\endhead
		\multicolumn{2}{r}{\textit{Continued on next page}} \\
		\endfoot
		\hline
		\endlastfoot
		
		$g$ & Acceleration due to gravity ($g = 9.8 ~\unit{kg/s^2}$ in this paper) \\
		$t$ & Time (counted from the initial time) \\
		$\rho_\text{air}$ & Density of air \\
		%$\rho_\text{oce}$ & Density of ocean water \\
		$m_\text{plane}$ & Total mass of the plane \\
		%$m_\text{box}$ & Total mass of the black box \\
		$A$ & Area of the wings (airfoils) of the plane \\
		$C_l$ & Lift coefficient of the wing in the air \\
		$C_d$ & Drag coefficient of the plane in the air \\
		$v_x$ & Velocity (with direction) of the plane in the positive $x$ direction \\
		$v_y$ & Velocity (with direction) of the plane in the positive $y$ direction \\
		$v_z$ & Velocity (with direction) of the plane in the positive $z$ direction \\
		%$v_\text{oce}$ & Velocity of the ocean current \\
		$S_x$ & Displacement of the plane in the positive $x$ direction \\
		$S_y$ & Displacement of the plane in the positive $y$ direction \\
		$S_z$ & Displacement of the plane in the positive $z$ direction \\
		$h$ & Height (Distance) of the plane above sea level \\
		%$V$ & The volume of the plane wreckage/black box \\
		%$k_x$ & The drag coefficient in $x$ direction underwater \\
		%$k_y$ & The drag coefficient in $y$ direction underwater \\
		%$k_z$ & The drag coefficient in $z$ direction underwater \\
		$\theta$ & Angle of attack of the wing (explained in \ref*{Terms def I})
							
	\end{longtable}  	
\end{center}
 

\subsubsection{Explanation of Terms and Phrases}\label{Terms def I}
\begin{enumerate}
%POC
	\item \textbf{POC}: Position of Contact is the exact position where the descending plane hits the surface of the ocean.
	
%rho
	\item \textbf{Density of Air}: The density of air $\rho_\text{air}$ changes with height $h$. From on-line database \cite{rho}, we could obtain the formula of $\rho_\text{air}$ in the  troposphere (always valid in this model, as discussed in \ref{Assumptions}).
	\begin{align*}
	& T = T_0 - Lh
	\\& P = P_0\cdot\left(1 - \dfrac{Lh}{T_0}\right)^{\frac{gM}{RL}}
	\\& \bm{\rho_\text{air}}\bm{ =} \dfrac{\bm{PM}}{\bm{1000\cdot RT}}
	\end{align*}
	where
	\begin{quote}
			$P_0 = 101325$  ~~~~~Sea level standard pressure, $\unit{Pa}$\\
			$T_0 = 288.15$ ~~~~~Sea level standard temperature, $\unit{K}$ \\
			$g = 9.80665$  ~~~~~~Gravitational constant, $\unit{m/s^2}$ \\
			$L = 6.5$   ~~~~~~~~~~~~~Temperature lapse rate, $\unit{K/km}$\\
			$R = 8.31432$  ~~~~~Gas constant, $\unit{J/ mol\cdot K} $ \\
			$M=28.9644$ ~~~ Molecular weight of dry air, $\unit{g/mol}$
 	
	\end{quote}
	By plotting $\rho_\text{air}$ against $h$ and applying a liner fit to the data point (with \textit{Origin}), we could find the following result:
	\begin{figure}[H]
		\centering
		\includegraphics[width = 5.0in]{linearfittorho.jpg}
		\caption{$\rho$ vs. $h$ (\textit{the real curve is the original data points, the dashed line is the fit line})}
	\end{figure}
	As can be judged from the graph and the related parameters, the linear fit works perfectly well. Thus, to simplify the calculations later in this model, we decide to approximate $\rho_\text{air}$ by the linear function of $h$ below (units are included):
	\begin{equation}
	\rho_\text{air} = -0.000080593\times h + 1.16661
	\end{equation}
	\textit{The detailed error analysis of this part will be included in \ref{error}.}
	
%angle of attack	
	\item \textbf{Angle of attack}: It represents the angle between the alignment of the wing and the horizontal direction in this case. It is denoted by $\theta$ in this model.
	
%lift
	\item \textbf{Lift force} \& \textbf{Lift coefficient}: \textbf{Lift force} is the force exerted by the air flow passing across the wing of the airplane that is perpendicular to the wing (hence the flying direction). According to \textit{aerodynamics}, the lift force could be calculated by:
	\begin{equation}
	F_\text{lift} = \dfrac{1}{2}\cdot  C_l\cdot\rho_\text{air}\cdot A\cdot v_\text{plane}^2
	\end{equation}
	where the coefficient $C_l$ is called the \textbf{lift coefficient}.
	\\The lift coefficient $C_l$ is usually calculated by experiments, and it's different with different wing shapes and angles of attack $\theta$. The relationship between $C_l$ and $\theta$ could be obtained by \textit{Fluent} and \textit{Gambit} \cite{airfoil} like below:
	\begin{figure}[H]
		\centering
		\includegraphics[height = 1.9in]{b707fluent.jpg}  \includegraphics[height = 2.0in]{b707linearfit.jpg}
		\caption{Model of wind tunnel and airfoil of Boeing 707, helping \textit{Fluent} yield the data, and the plot depicting the relation between $C_l$ and $\theta$ for Boeing 707}
		
		\includegraphics[height = 1.9in]{b737fluent.jpg}  \includegraphics[height = 2.0in]{b737linearfit.jpg}
		\caption{Model of wind tunnel and airfoil of Boeing 737, helping \textit{Fluent} yield the data, and the plot depicting the relation between $C_l$ and $\theta$ for Boeing 737}
	\end{figure}

	
%drag
	\item \textbf{Drag force \& Drag coefficient}: \textbf{Drag force} is the force that impedes the plane from moving forward. It usually counteracts with the propulsion of the plane to give it a steady airspeed. In this model, we assume that the drag force always acts horizontally and opposite to the flying direction. The drag force obeys the following function according to \textit{aerodynamics}:
	\begin{equation}
	F_\text{drag} = \dfrac{1}{2} \cdot C_d \cdot \rho_\text{air} \cdot A\cdot v_\text{plane}^2
	\end{equation}
	where $C_d$ is called the \textbf{drag coefficient}.
	\\In general, we would like to take $C_d = 0.08$ in our model.
	
%airplane database
	\item \textbf{Airplane database}: Different airplanes have different feature parameters. We would like to select the following key parameters to characterize an airplane; in the mean time, we would like to build a preliminary database of airplanes characterized by the key parameters \cite{plane database}.
	\\Note that the \textit{lift coefficient} $C_l$ could be calculated by \textit{Fluent}. Due to time limit, we were not able to simulate the \textit{lift coefficients} for each airplane.
	% Table generated by Excel2LaTeX from sheet 'Sheet2'
	\begin{center}
		\begin{longtable}{ccccccccc}
			\caption{Airplane Database \cite{plane database}}\\
				\hline
			Airplane  &  \tabincell{c}{Largest\\ Takeoff \\Mass (\unit{kg})} & \tabincell{c}{Length \\(\unit{m})} & \tabincell{c}{Width\\ (with \\airfoils)\\ (\unit{m})} &\tabincell{c}{Height\\ (\unit{m})} & \tabincell{c}{Seat\\ number} & \tabincell{c}{Airfoil\\ Areas ($\unit{m^2}$)} & \tabincell{c}{Lift\\ Coefficient \\($\theta = 5^\circ$)} \\
				\hline\hline
				\endfirsthead
				\multicolumn{8}{c}%
				{\tablename\ \thetable\ -- \textit{Continued from previous page}} \\
				\hline
				Airplane  &  \tabincell{c}{Largest\\ Takeoff \\Mass (\unit{kg})} & \tabincell{c}{Length \\(\unit{m})} & \tabincell{c}{Width\\ (with \\airfoils)\\ (\unit{m})} &\tabincell{c}{Height\\ (\unit{m})} & \tabincell{c}{Seat\\ number} & \tabincell{c}{Airfoil\\ Areas ($\unit{m^2}$)} & \tabincell{c}{Lift\\ Coefficient \\($\theta = 5^\circ$)} \\
				\hline\hline
				\endhead
				\multicolumn{8}{r}{\textit{Continued on next page}} \\
				\endfoot
				\hline
				\endlastfoot
		
		
		\tabincell{c}{Boeing\\ 707} &     150590 &      44.07 &       39.9 &      12.93 &        179 &      268.6 &   0.54216         \\
	
		\tabincell{c}{Boeing\\ 737-900} &      85130 &       42.1 &       35.7 &       12.5 &        215 &        125 &     0.71470       \\
		
		\tabincell{c}{Boeing\\ 747-400} &     396890 &       70.6 &       64.4 &       19.4 &        416 &        511 &  $\cdots $        \\
	
		\tabincell{c}{Boeing\\ 747-8} &     439985 &       76.4 &       68.5 &       19.3 &        467 &        554 &   $\cdots $         \\

		\tabincell{c}{Boeing\\ 777-300} &     299370 &       73.9 &       60.9 &       18.5 &        368 &      427.8 &  $\cdots $          \\
	
		\tabincell{c}{Boeing\\ 787-8} &     247000 &         63 &         60 &      16.92 &        290 &        358 &   $\cdots $         \\
	
		\tabincell{c}{Airbus \\ A320-100} &      77000 &      37.57 &       34.1 &      11.76 &        180 &        143 &  $\cdots $          \\
	
		\tabincell{c}{Airbus \\A330-300} &     230000 &       63.6 &       60.3 &      16.85 &        295 &      361.6 &  $\cdots $          \\
	
		\tabincell{c}{Airbus \\A380} &     590000 &         72 &       79.8 &      24.01 &        653 &        845 &  $\cdots $          \\
	
		\tabincell{c}{Antonov\\ An-225} &     640000 &         84 &      88.74 &       18.1 &         70 &       1069 &  $\cdots $          \\
	
		
		\end{longtable}  
	\end{center}
	
	
	
	
\end{enumerate}

\subsubsection{Modeling}
First, by placing the plane in a \textit{2D-Cartesian coordinate system} centered at the location where the technical failure occurs, we analyze the situation with Newton's laws in dynamics and generate the free-body diagram of the plane during the descending:
\begin{figure}[H]
	\centering
	\includegraphics[width = 3.2in]{freebodydiagrammodeli.jpg}
	\caption{Free-body diagram of the plane in descending}
\end{figure}
\noindent Note that we have obtained the expressions of $F_\text{lift}$ and $F_\text{drag}$ in \ref{Terms def I}, and $m_\text{plane}$ should be substituted with practical values. 
\\To calculate the full trajectory of the descending process, we will also have to specify the initial airspeed of the airplane just before the technical failure as ${v_x}_0$, the initial vertical speed ${v_y}_0$, and the initial height $h_0$.
\\Then, we could obtain the following equations representing the motion of the airplane during the descending:
\[
\left\{
\begin{aligned}
 m_\text{plane}\dfrac{dv_x}{dt} &= -\dfrac{1}{2}(C_d + C_l\sin\theta)\cdot(-0.000080593\times h + 1.16661)Av_x^2
\\ m_\text{plane}\dfrac{dv_y}{dt} &= m_\text{plane}\cdot g - \dfrac{1}{2}C_l\cos\theta\cdot(-0.000080593\times h + 1.16661)Av_x^2
\\ \dfrac{dh}{dt} &= -v_y
\end{aligned}
\right.
\]
or
\[
\left\{
\begin{aligned}
m_\text{plane}\dfrac{d^2s_x}{dt^2} &= -\dfrac{1}{2}(C_d + C_l\sin\theta)\cdot(-0.000080593\times h + 1.16661)A\left(\dfrac{ds_x}{dt}\right)^2
\\ m_\text{plane}\dfrac{d^2s_y}{dt^2} &= m_\text{plane}\cdot g - \dfrac{1}{2}C_l\cos\theta\cdot(-0.000080593\times h + 1.16661)A\left(\dfrac{ds_x}{dt}\right)^2
\\ \dfrac{dh}{dt} &= -\dfrac{ds_y}{dt}
\end{aligned}
\right.
\]
By applying the initial conditions mentioned above, we could solve both systems of \textit{differential equations} with \textit{MATLAB} to obtain $v_x$, $v_y$, $s_x$, and $s_y$.
\\Next, by solving the equation
\begin{align*}
s_y = h_0
\end{align*}
we would obtain the time $t$ for the plane to reach \textit{POC}, thus obtain:
\begin{quote}
	$v_\text{horizontal}$ ~~~ The horizontal speed of the airplane at \textit{POC};\\
	$v_\text{vertical}$ ~~~~~~~~ The vertical speed of the airplane at \textit{POC};\\
	$s_\text{horizontal}$ ~~~~~ The horizontal distance the plane traveled during  the descending.
\end{quote}
With the information above, we could now vaguely determines the \textit{POC} from the original flying path. 
\\{\it Due to the uncertainty of a practical case, the initial speed ${v_x}_0$ could vary greatly, and the predicted location in tern lies in a relative broader area rather than an exact point, which will be illustrated in \ref{Phase I}.}









%Model II: Sinking Model
\subsection{Model II: Sinking Model}
By this time, the plane has passed \textit{POC} and has turned into pieces of wreckage. {\it Some of the wreckage will directly sink to the bottom of the ocean, including the black boxes; some of it will remain floating on the surface of the ocean  and wait to be discovered by search teams.}
\\This model deals with the situation where the wreckage and the black boxes are sinking to the bottom of the ocean and finally rest on the seabed. {\it Particularly, we would like to consider the motion of a black box, which is more convenient to calculate. The motion of other forms of plane wreckage that will sink to the bottom of the ocean are mostly of very high similarities.} 
\subsubsection{Notations}
\begin{center}
	\begin{longtable}{cl}
		\caption{Notations (Model II)}\\
		\hline
		Symbol &  Meaning \\
		\hline\hline
		\endfirsthead
		\multicolumn{2}{c}%
		{\tablename\ \thetable\ -- \textit{Continued from previous page}} \\
		\hline
		Symbol &    Meaning \\
		\hline\hline
		\endhead
		\multicolumn{2}{r}{\textit{Continued on next page}} \\
		\endfoot
		\hline
		\endlastfoot
		
		$g$ & Acceleration due to gravity ($g = 9.8 ~\unit{kg/s^2}$ in this paper) \\
		$t$ & Time (counted from the initial time, reset in different sub-models) \\
		%$\rho_\text{air}$ & Density of air \\
		$\rho_\text{oce}$ & Density of ocean water \\
		%$m_\text{plane}$ & Total mass of the plane \\
		$m$ & Total mass of the black box \\
		%$S$ & Area of the wings of the plane \\
		%$C_l$ & Lift coefficient of the wing in the air \\
		%$C_d$ & Drag coefficient of the plane in the air \\
		${v_b}_{x}$ & Velocity (with direction) of the black box in the positive $x$ direction \\
		${v_b}_{y}$ & Velocity (with direction) of the black box in the positive $y$ direction \\
		${v_b}_{z}$ & Velocity (with direction) of the black box in the positive $z$ direction \\
		$v_\text{oce}$ & Velocity of the ocean current \\
		${S_b}_{x}$ & Displacement of the black box in the positive $x$ direction \\
		${S_b}_{y}$ & Displacement of the black box in the positive $y$ direction \\
		${S_b}_{z}$ & Displacement of the black box in the positive $z$ direction \\
		$D$ & Depth of the ocean \\
		$V$ & The volume of the plane wreckage/black box \\
		$k_x$ & The motion coefficient in $x$ direction underwater \\
		$k_y$ & The motion coefficient in $y$ direction underwater \\
		$k_z$ & The motion coefficient in $z$ direction underwater \\
		
	\end{longtable}  	
\end{center}

\subsubsection{Explanation of Terms and Phrases}\label{Terms def ii}
\begin{enumerate}
%wreckage
	\item \textbf{Wreckage}: Directly after passing \textit{POC}, the plane will immediately turn into pieces. Those that have a larger density than sea water will sink to the bottom of the ocean, while the other pieces will remain floating on the surface of the ocean. 
	
%main wreckage
	\item \textbf{Main wreckage}: Pieces of large size and representing the main body of the plane which rest on the seabed are called \textbf{main wreckage}.
	
%black box
	\item \textbf{Black box}: It is also known as \textit{flight recorder}, which records the almost all the flying data during the flight. In this model, we would like to approximate the shape of the black box as a \textit{cuboid}, which will experience much regular resistance underwater.
	
%drag force
	\item \textbf{Drag force (underwater)}: When a moving object is placed underwater, it will inevitably experience a resistance \textit{opposite} to its moving direction, known as the \textbf{drag force} $f_\text{drag}$. As discussed before, we have approximated the shape of a black box as a cuboid, which will experience drag forces that is direct proportional to its speed according to \textit{hydrodynamics}. That is,
	\begin{align*}
	f_\text{drag} = k_\text{drag}\cdot v
 	\end{align*}
	where $k_\text{drag}$ is generally known as the drag coefficient, and $v$ is the \textit{relative speed} of the object with respect to that of the ocean current.
	
%push force
	\item \textbf{Push force (underwater)}: When an object is placed underwater, it will also experience a push force  exerted by the ocean current (\textit{along} the direction of ocean current movement). Usually this force is very hard to predict and measure, due to the unpredictable nature of an ocean. To simplify the model within reasonable extent, we approximate this force by a linear function like above, and assume that the push coefficient $k_\text{push}$ is everywhere the same:
	\begin{align*}
	f_\text{push} = k_\text{push} \cdot v
	\end{align*}
	Again, $v$ is the \textit{relative speed} of the object with respect to that of the ocean current.
	
%motion force	
	\item \textbf{Motion force}: It is simply a combination of the \textit{push force} and \textit{drag force} in $x$, $y$, and $z$ directions. It is denoted by $f_i$ in this model. While \textit{drag force} is always present in this model, \textit{push force} only exists in the direction of the ocean current movement. Therefore,
	\begin{quote}
		When there is \textit{push force} present, 
		\begin{align*}
		f_i = (k_\text{drag} - k_\text{push})\cdot v = \bm{k_i}v
		\end{align*}
		When there is NO \textit{push force} present,
		\begin{align*}
		f_i = k_\text{drag}\cdot v = \bm{k_i}v
		\end{align*}
	\end{quote}
	where $\bm{k_i}$ is called the \textbf{motion coefficient}.
	\\We also assume that the direction of the \textit{motion force} is initially \textit{opposite} to the object's motion (\textit{along} the direction of the \textit{drag force}).

\end{enumerate}


\subsubsection{Modeling}
From \textbf{Model I}, we could find the horizontal and vertical speed (${v_b}_{y0}$ and ${v_b}_{z0}$) of the wreckage/black box directly after passing \textit{POC}, which are equal to the plane's horizontal and vertical speed ($v_\text{horizontal}$ and $v_\text{vertical}$). 
\\This time we place the black box in a \textit{3D-Cartesian coordinate system} centered at \textit{POC}. Now, consider three scenarios:
\begin{quote}
	1. The direction of ${v_b}_{y0}$ is the same as that of $v_\text{oce}$ (assume that the \textit{motion force} is too weak to slow the object down to $v_\text{oce}$ before it hit seabed);
	\\2. The direction of ${v_b}_{y0}$ is perpendicular to that of $v_\text{oce}$.
	\\3. The angle between the direction of ${v_b}_{y0}$ and $v_\text{oce}$ is $\alpha$.
\end{quote}
First, consider the first two scenarios:
\begin{figure}[H]
	\centering
	\includegraphics[width = 2.9 in]{freebodydiagrammodelii2d.jpg}
	\includegraphics[width = 2.9 in]{freebodydiagrammodelii3d.jpg}
	\caption{The diagram on the left corresponds to scenario 1; the one on the right corresponds to scenario 2.}
\end{figure}
\noindent For scenario 1, we could have the following system of differential equations depicting the motion of the black box once it's underwater:
\[
\left\{
\begin{aligned}
& m\dfrac{d{v_b}_y}{dt} = k_y({v_b}_y - v_\text{oce})
\\& m\dfrac{d{v_b}_z}{dt} = mg - \rho g V - k_z {v_b}_z
\end{aligned}
\right.
\]
or
\[
\left\{
\begin{aligned}
& m\dfrac{d^2{S_b}_y}{dt^2} = k_y(\dfrac{d{S_b}_y}{dt} - v_\text{oce})
\\& m\dfrac{d^2{S_b}_z}{dt^2} = mg - \rho g V - k_z \dfrac{d{S_b}_z}{dt}
\end{aligned}
\right.
\]
where $m$, $k_y$, $k_z$, $\rho$, $V$, and $g$ are specified. Also note that ${v_b}_y - v_\text{oce}$ is the relative speed of the black box with respect to the ocean current.
\\~
\\For scenario 2, we could have the following system of differential equations depicting the motion of the black box once it's underwater:
\[
\left\{
\begin{aligned}
& m\dfrac{d{v_b}_x}{dt} = k_x({v_b}_x - v_\text{oce})
\\& m\dfrac{d{v_b}_y}{dt} = - k_y{v_b}_{y}
\\& m\dfrac{d{v_b}_z}{dt} = mg - \rho g V - k_z {v_b}_z
\end{aligned}
\right.
\]
or
\[
\left\{
\begin{aligned}
& m\dfrac{d^2{S_b}_x}{dt^2} = k_x(\dfrac{d{S_b}_x}{dt} - v_\text{oce})
\\& m\dfrac{d^2{S_b}_y}{dt^2} = -k_y\dfrac{d{S_b}_y}{dt}
\\& m\dfrac{d^2{S_b}_z}{dt^2} = mg - \rho g V - k_z \dfrac{d{S_b}_z}{dt}
\end{aligned}
\right.
\]
where $m$, $k_y$, $k_z$, $\rho$, $V$, and $g$ are specified. Also note that ${v_b}_y - v_\text{oce}$ is the relative speed of the black box with respect to the ocean current.
\\~
\\Both the model above will be solved by \textit{MATLAB} once the initial conditions ${v_b}_{x0}$, ${S_b}_{x0}$, ${v_b}_{y0}$, ${S_b}_{y0}$, ${v_b}_{z0}$, and ${S_b}_{z0}$ are provided.
\\~
\\Now, for scenario 3, we would have the following free-body diagram:
\begin{figure}[H]
	\centering
	\includegraphics[width = 3.3 in]{freebodydiagrammodelii33d.jpg}
	\caption{Free-body diagram of scenario 3}
\end{figure}
\noindent Therefore, the equations of motion could be represented by the system of differential equations below: 
\[
\left\{
\begin{aligned}
& m\dfrac{d{v_b}_x}{dt} = k_x({v_b}_x - v_\text{oce}\sin\alpha)
\\& m\dfrac{d{v_b}_y}{dt} = -k_y({v_b}_y - v_\text{oce}\cos\alpha)
\\& m\dfrac{d{v_b}_z}{dt} = mg - \rho g V - k_z {v_b}_z
\end{aligned}
\right.
\]
or
\[
\left\{
\begin{aligned}
& m\dfrac{d^2{S_b}_x}{dt^2} = k_x(\dfrac{d{S_b}_x}{dt} - v_\text{oce}\sin\alpha)
\\& m\dfrac{d^2{S_b}_y}{dt^2} = -k_y(\dfrac{d{S_b}_y}{dt} - v_\text{oce}\cos\alpha)
\\& m\dfrac{d^2{S_b}_z}{dt^2} = mg - \rho g V - k_z \dfrac{d{S_b}_z}{dt}
\end{aligned}
\right.
\]
Again, by providing the initial conditions ${v_b}_{x0}$, ${S_b}_{x0}$, ${v_b}_{y0}$, ${S_b}_{y0}$, ${v_b}_{z0}$, and ${S_b}_{z0}$, the exact solution could be solved by \textit{MATLAB}. Here we present the symbolic solution of scenario 3 in terms of the distance traveled $S_i$:
\[
\left\{
\begin{aligned}
& {S_b}_x = v_\text{oce}\sin\alpha \cdot t - \dfrac{mv_\text{oce}\sin\alpha}{k_x}\cdot e^{\frac{k_x}{m}t} + \dfrac{mv_\text{oce}\sin\alpha}{k_x}
\\& {S_b}_y = v_\text{oce}\cos\alpha \cdot t - \dfrac{m({v_b}_{y0} - v_\text{oce}\cos\alpha)}{k_y}\cdot e^{\frac{k_x}{m}t} + \dfrac{m({v_b}_{y0} - v_\text{oce}\cos\alpha)}{k_y} 
\\& {S_b}_{z} = \dfrac{g(m-\rho V)}{k_z}t + \left[{v_b}_{z0} - \dfrac{g(m - \rho V)}{k_z}\right]\dfrac{m}{k_z}\cdot e^{-\frac{k_z}{m}t} + \dfrac{mg(m-\rho V)}{k_z^2} - \dfrac{m{v_b}_{z0}}{k_z}
\end{aligned}
\right.
\]
This way, we will be able to know the exact location of the black box at any time before it reaches the seabed (hence the main wreckage).
\\~\\We also present a database of the \textit{depth} of the different oceans in order to make our model more applicable to real-life scenarios:
\begin{center}
	\begin{longtable}{ccc}
		\caption{Ocean Depth Database }\\
		\hline
		Ocean & Largest Depth (m) & Average Depth (m) \\
		\hline\hline
		\endfirsthead
		\multicolumn{2}{c}%
		{\tablename\ \thetable\ -- \textit{Continued from previous page}} \\
		\hline
		Ocean & Largest Depth (m) & Average Depth (m) \\
		\hline\hline
		\endhead
		\multicolumn{2}{r}{\textit{Continued on next page}} \\
		\endfoot
		\hline
		\endlastfoot
		
		Pacific &11.34 & 4028 \\
		Atlantic & 9218 & 3629 \\
		Indian & 9074 & 3872 \\
		Arctic & 5527 & 1225\\
		
	\end{longtable}
\end{center}
$$$$


%Model III: Searching Model
\subsection{Model III: Searching Model}
\subsubsection{Notations}
\begin{center}
	\begin{longtable}{cl}
		\caption{Notations (Model III)}\\
		\hline
		Symbol &  Meaning \\
		\hline\hline
		\endfirsthead
		\multicolumn{2}{c}%
		{\tablename\ \thetable\ -- \textit{Continued from previous page}} \\
		\hline
		Symbol &    Meaning \\
		\hline\hline
		\endhead
		\multicolumn{2}{r}{\textit{Continued on next page}} \\
		\endfoot
		\hline
		\endlastfoot
		
		$p_i$ & Priority of the $i$th area\\
		
	\end{longtable}  	
\end{center}


\subsubsection{Explanation of Terms and Phrases}
\begin{enumerate}
	
%region/zone
	\item \textbf{Region} \& \textbf{Zone}: In our search plan, the possible crashing \textbf{region} is a circle centered at the point where it lost communication with ground. The radius of the circle is the distance between the center and the farthest possible \textit{POC} (\textit{according to the statistics of the past air crash , the farthest possible POC is around 100 \unit{km}}).
	\\Depending on the practical situation, the \textit{region} could be divided into several \textbf{zones}. Certain numbers of search planes will be assigned to each \textit{zone.} The division obeys the following two general rules:
	\begin{quote}
		1. Each \textit{zone} tends to have the shape of a circle;
		\\2. Differences in area between \textit{zones} should be kept small.
	\end{quote}
	
%POD
	\item \textbf{POD}: \textbf{Position of Discovery} is the position where the search teams found floating wreckage, bodies or survivors.
	
%priority
	\item \textbf{Priority}: Each \textit{zone} in the region is assigned with a parameter $p_i$ representing the searching \textbf{priority}. The higher the possibility of spotting floating wreckage or survivors, the larger $p_i$ should be, and the more ``important'' the \textit{zone} is. Meanings of different values of $p_i$ are listed below:
	\begin{center}
		\begin{longtable}{cll}
			\caption{Priority (Model III)}\\
			\hline
			Value of Priority $p$ &  Possibility of Discovery & Searching Pattern \\
			\hline\hline
			\endfirsthead
			\multicolumn{3}{c}%
			{\tablename\ \thetable\ -- \textit{Continued from previous page}} \\
			\hline
			Value &  Priority $p$  & Searching Pattern\\
			\hline\hline
			\endhead
			\multicolumn{3}{r}{\textit{Continued on next page}} \\
			\endfoot
			\hline
			\endlastfoot
			
			$0$ & Impossible & No need for searching \\
			$1$ & Possible & Regular searching \\
			$2$ & Likely & Extra-regular searching \\
			$3$ & Probable & Intensive searching \\
			$4$ & Highly-probable & Advanced searching \\
			$5$ & Certain & All-out searching \\
		\end{longtable}
	\end{center}
	There are two main factors that contribute to the \textit{priority} $p$ of each \textit{zone}:
	\begin{quote}
		1. The area of the \textit{zone}.
		\\2. The \textit{possibility of discovery} of the \textit{zone}, which is related to:
		\begin{quote}
			a. The crashing direction of the plane (usually the same as its original course). This determines the initial searching pattern.
			\\b. The discovery of floating objects. This determines the searching pattern afterwards, which is discussed later in the text.
		\end{quote}
	\end{quote}
	
%plane devices
	\item \textbf{Planes} \& \textbf{Devices}: In our model, there are 3 sorts of searching planes and 3 sorts of rescue and salvage equipment, which are listed in Table \ref{planes and devices}. 	\textit{The searching resource is limited.} Recall that by reinstalling new devices onto the search planes, the function of the plane can change, and hence the Label.
	\begin{center}
		\begin{longtable}{lll}
			\caption{Planes \& Devices}\label{planes and devices}\\
			\hline
			Label &  Function & Device \\
			\hline\hline
			\endfirsthead
			\multicolumn{3}{c}%
			{\tablename\ \thetable\ -- \textit{Continued from previous page}} \\
			\hline
			Label &  Function & Device \\
			\hline\hline
			\endhead
			\multicolumn{3}{r}{\textit{Continued on next page}} \\
			\endfoot
			\hline
			\endlastfoot
			
			Plane \textbf{P1} & Searching survivors & Thermal imaging devices \\
			Plane \textbf{P2} & Searching floating wreckage & Remote sensing devices \\
			Plane \textbf{P3} & Searching main wreckage & Sonar \\
			Helicopter \textbf{H1} & Rescue and Salvage & --\\
			Ship \textbf{S1} & Rescue and Salvage & -- \\
			Submarine \textbf{S2} & Rescue and Salvage & --\\
			
		\end{longtable}
	\end{center}

%ref circle & distance
	\item \textbf{Reference circle} \& \textbf{Reference distance}: \textbf{Reference circle} determines the general boundaries that divides the whole \textit{Region}; while the \textit{reference distance} is twice the  distance between the position where the plane lost communication with the ground and the nearest possible \textit{POC}.
	
%shift
	\item \textbf{Shift}: To enhance the quality and effectiveness of the searching and rescuing operation, we arrange 8 \textbf{shifts} during a day. In this model, we neglect the time of \textit{shift} changing. Each shift lasts around 3 hours.
		
	
\end{enumerate}

\subsubsection{Modeling}
\subsubsection*{Resource Arrangement}
The number of planes sent to each \textit{zone} should be directly proportional to the \textit{priority} $p_i$ of each \textit{zone}. The division of \textit{zones} and the assignment of priorities will be discussed in the next section.
\\It is also noteworthy that the motion of the ocean surface current can not be neglected, as the search operation may take days or even months. Thus the \textit{POD}s of floating wreckage, bodies or survivors may not be their initial {\it POC}s. Therefore, we have to inform all the pilots that the searching \textit{zone} they are responsible for is moving along with the ocean surface current. In other words, one's \textit{zone} is not of absolute latitude \& longitude.



\subsubsection*{Search Plans}
In order to deal with various and even unpredicted situations, we construct 5 search and rescue plans. Each plan has its own division of the whole \textit{region} and arrangements of the searching resources. {\it It is assumed that only Plane P1 and P3 could work during night, hence can be sent for mission.} Detailed instructions are listed below:
\begin{itemize}
	\item Plan A: 
	\\This plan is based on the probable direction of the crashing of the plane (usually the same as the original flight path). 
	\begin{itemize}
		\item The proportion of Plane P1, P2, and P3 is $4:1:1$. 
		\item Division of the \textit{region} and the assignment of \textit{priority} are shown in Figure \ref{plan A}. Priority is the number at the center of each \textit{zone}.
		\item The \textit{reference circle} in dashed curve is centered at the lost-of-communication point and has a radius of the \textit{reference distance}. 
		\item For plane P1, the \textit{zone} at the center of the \textit{region} is shaped as the inscribed pentagon of the \textit{reference circle}. All other boundaries intersect at the center of the \textit{reference circle}.
		
	\end{itemize}
	\begin{figure}[H]
		\centering
		\includegraphics[width = 4.6in]{planA.jpg}
		\caption{The arrangement for plane P1 is on the left, which searches for survivors.  ~~ The arrangement of P2 and P3 are the same and is shown on the right. Note that the \textit{priority} is higher in the flying direction.}\label{plan A}
	\end{figure}



	\item Plan B: 
	\\This plan is executed once survivors are found. 
	\begin{itemize}
		\item The proportion of planes changes to $6:1:1$.
		\item For plane P1, new division of \textit{region} and assignments of \textit{priority} is shown in Figure \ref{planBP1}. The \textit{reference circle} in dashed lines is centered at the point where the survivors are found and the radius is the  \textit{reference distance}. 
		\item The helicopters H1, ships S1, and submarines S2 should also be sent to the \textit{zone} with the highest \textit{priority} immediately.
		\item For plane P2 and P3,  new assignments of priority is shown in Figure \ref{planBP2P3}. Note that the division remains the same as plan A, but we change the \textit{priority} of the \textit{zone} where the survivors are found to 2 and the rest to 1.
		
	\end{itemize}
	\begin{figure}[H]
		\centering
		\includegraphics[width = 4.6in]{planBP1.jpg}
		\caption{For plane P1, the division of the region differs as the 3 different types of location of the POD. The 3 types are that \textit{reference circle} includes \textit{the center of the area, the edge of the area or none of above.} The \textit{priority} of the \textit{zone} with the POD is 4 and the rest is 2. Note that in the right most figure, the pentagon at the center  of the \textit{region} is congruent to the one just below it.}\label{planBP1}
		\end{figure}
		\begin{figure}[H]
		\centering
		\includegraphics[width = 4.6in]{planBP2P3.jpg}
		\caption{The arrangement for plane P2 is on the left, the \textit{priority} of the \textit{zone} including the POD is 2 and the rest are 1. ~~ The arrangement for plane P3 is on the right, we calculate the possible spot of main wreckage and change the \textit{priority} of the \textit{zone} including this spot to 2 and the rest to 1.}\label{planBP2P3}
	\end{figure}
	
	\item Plan C: 
	\\Plan C is active when floating wreckage or oil spill is spotted.
	\begin{itemize}
		\item The proportion changes to $3:1:2$.
		\item This time, plane P1 and P2 share the same division of the area and are assigned according to the value of \textit{priority} shown in Figure \ref{planC}.
		\item For plane P3, we first use the location of the floating wreckage and the speed of ocean surface current to calculate the possible  \textit{POC}. Then we will estimate the area of main wreckage using \textbf{model II}.
	\end{itemize}
	\begin{figure}[H]
		\centering
		\includegraphics[width = 4.6in]{planC.jpg}
		\caption{For plane P1 and P2, the division is shown on the left (note that P1 and P2 this time share the same division). The \textit{priority} of the \textit{zone} including the POD is 4 and that of the \textit{zones} next to it change toward ``2'' by ``1'' (for example, 4 to 3, 3 to 2, 1 to 2). The rest (if any) is changed toward ``1'' by ``1'' (for example, 3 to 2, 2 to 1). The arrangement for  plane P3 is on the right. The \textit{priority} of the \textit{zone} with the possible point of main wreckage is 3 and the rest are 1.}\label{planC}
	\end{figure}
	
	
	\item Plan D: 
	\\When the floating wreckage found earlier indicates that the plane has gone through disintegration or explosion in the air, plan D will be carried out.
	\begin{itemize}
		\item In such circumstances, the massive discovery of wreckage shows that there is a very high possibility for the main wreckage and survivors to be around.
		\item The proportion changes to $1:3:3$.
		\item The new division and assignment of \textit{priority} is shown in Figure \ref{planD}.
	\end{itemize}
	\begin{figure}[H]
		\centering
		\includegraphics[width = 5.0in]{planD.jpg}
		\caption{Under such circumstances, it is almost certain that all of the wreckage and survivors are in this \textit{zone}. Therefore, the \textit{priority} of the \textit{zone} is 5 and the rest is 1.}\label{planD}
	\end{figure}
	
	
	\item Plan E:  
	\\Plan E is activated when the main wreckage is found ``luckily''.
	\begin{itemize}
		\item The proportion of plane P1 and P2 is 2:1. Plane P3 is inactive.
		\item The new division and assignment of \textit{priority} is shown in Figure \ref{planE}.
		\item The \textit{zone} with the highest \textit{priority} should be determined through calculation with speed of ocean surface current taken into consideration.
	\end{itemize}
	\begin{figure}[H]
		\centering
		\includegraphics[width = 5.0in]{planE.jpg}
		\caption{For plane P1 and P2, they again share the same division of the \textit{region}. First, we calculate the possible point of floating wreckage and survivors. Then the \textit{priority} of the \textit{zone} including this point is assigned 4 and those of \textit{zones} next is 2. The rest are 1.}\label{planE}
	\end{figure}
		

\end{itemize}



\subsubsection*{Searching Algorithm}
\begin{itemize}
	\item When the loss of communication is confirmed, nearest and farthest possible \textit{POC}s should be calculated using \textbf{model I}. Afterwards, the general \textit{region} is determined.
	
	\item \textbf{Initial Status}:
	\\\fbox{\textbf{Plan A}} is active. Helicopters, ships, and submarines  will stand by.
	
	\item Something discovered:
	
	\begin{itemize}
		\item Survivors:
		\\ \fbox{\textbf{Plan B}} is active. Helicopters, ships, and submarines are sent for rescue operations as well as search assistance.
		
		\begin{itemize}
			\item Small number of survivors:
			\\The \textit{priority} of the \textit{zone} will be changed according to \fbox{\textbf{Plan B}}.
			
			\item Relatively large number of survivors  as well as sign of main wreckage:
			\\All efforts will be put into the \textit{zone}. \textit{Priority} of the current \textit{zone} rises to 5 and the rest drop to 1.
		\end{itemize}
		
		\item Floating wreckage or oil spill:
		\\Further searching is needed to determine whether the plane has went through disintegration or explosion.
		\begin{itemize}
			\item Just a little wreckage:
			\\\fbox{\textbf{Plan C}} is active. Helicopters, ships, and submarines stand by.
			\item Disintegration or explosion confirmed:
			\\\fbox{\textbf{Plan D}} is active. All the helicopters, ships and submarines are sent for rescue and salvage.
		\end{itemize}
		
		\item Main wreckage:
		\\\fbox{\textbf{Plan E}} is active. Helicopters, ships and submarines for salvage are sent out for mission.
		
	\end{itemize}
	\item No discoveries in a period of time:
	\\All search planes are \textit{reset} to the \textbf{initial status} when there are no further discoveries for 3 \textit{shifts} in the previous status of searching.
	
	\item As time goes by, the numbers of each plane follows the rules below:
	\begin{itemize}
		\item 	Every 5 days, half of the plane P1 are changed to plane P2 or P3.
		\item Every 10 days, $\dfrac{1}{3}$ of the planes are dismissed.
	\end{itemize}
\end{itemize}











\section{Simulation/Model Testing}\label{Test}
Now, we apply our model to a real-life example:
Suppose that a Boeing 737-900 passenger airplane is flying across the Atlantic ocean from Paris to New York. The information about the flying status is listed below:
\begin{center}
	\begin{longtable}{ll}
		\caption{Simulation of Boeing 737-900 Flying Status}\\
		\hline
		Parameter & Value\\
		\hline\hline
		\endfirsthead
		\multicolumn{2}{c}%
		{\tablename\ \thetable\ -- \textit{Continued from previous page}} \\
		\hline
		Parameter & Value\\
		\hline\hline
		\endhead
		\multicolumn{2}{r}{\textit{Continued on next page}} \\
		\endfoot
		\hline
		\endlastfoot
		
		Initial height $h_0$ &  $10000~\unit{m}$
		\\Initial horizontal speed ${v_x}_0$ & $ 243.33 ~\unit{m/s}$
		\\Angle of attack $\theta$ & $ 5^\circ$
		\\Initial vertical speed ${v_y}_0$ & $ 0~ \unit{m/s}$
		\\Lift coefficient ($\theta = 5^\circ$) $C_l$ & $0.71470$
		\\Drag coefficient $C_d$ & $ 0.08$
		\\Wing area $A$ &$ 125 ~\unit{m^2}$
		\\Total mass $m_\text{plane} $ & $ 85130 ~\unit{kg}$
		\\Speed of Gulf Stream  $v_\text{oce}$ & $ 0.3028~\unit{m/s}$
		\\Mass of black box $m$ & $ 10~\unit{kg}$
		\\Volume of black box $V$ & $ 0.015~\unit{m^3}$
		\\The angle between the ocean current and the direction of crashing $\alpha$ & $ 30^\circ$
		\\Motion coefficient in each direction & $k_x = k_y = k_z = 0.6$
		\\Density of ocean water $\rho_\text{oce}$ & $ 1010~\unit{kg/m^3}$\\
		
	\end{longtable}
\end{center}
Unfortunately, the airplane suffered a technical failure above the Gulf Stream (\textbf{38$^\circ$43'31.9''N, 65$^\circ$19'40.6''W}) and began descending (no disintegration or explosion occurred), which could be analyzed by \textbf{Model I}:

\subsection{Phase I: Descending}\label{Phase I}
According to \textit{MATLAB}, we have the following data for the plane at \textit{POC}:
\begin{quote}
	Falling time: $t = 100.2 ~\unit{s}$
	\\Horizontal displacement: 16557 ~\unit{m}
	\\Horizontal speed: $v_\text{horizontal} = 99.0 ~\unit{m/s}$
	\\Vertical speed: $v_\text{vertical} = 257.8 ~\unit{m/s}$
\end{quote}
Also, we could generate a graph depicting the airplane's flying trajectory by \textit{MATLAB}. It is shown in Figure \ref{trajector}.
\begin{figure}[H]
	\centering
	\includegraphics[width = 6.5 in]{matlabphasei.jpg}
	\caption{Flying trajectory}\label{trajector}
\end{figure}
\subsection{Phase II: Sinking}\label{Phase II}
By using the data from \textbf{Phase I} as initial conditions, we could calculate the following information about the black box by \textbf{Model II} in \textit{MATLAB}:
\begin{quote}
	Sinking time: $t = 168.4$ \unit{s}
	\\Displace in $x$ direction: 127 \unit{m}
	\\Displacement in $y$ direction (crashing direction): 1209 \unit{m}
\end{quote}
$$$$
By now, we have obtained the total displacement of the black box in space. Therefore, we could pinpoint the approximated location of \textit{POC} and the position that lies the black box (hence the main wreckage) on \textit{Google Maps}.
\begin{figure}[H]
	\centering
	\includegraphics[width = 5.6 in]{google.jpg}
	\caption{location on \textit{Google}\cite{google}}
\end{figure}

\subsection{Phase III: Searching}\label{Phase III}
We coded a C++ program in \textit{Visual Studio} in order to simulate the process of searching. Random numbers are generated to determine whether the discovery took place. The numbers of planes determined by the \textit{priority} influence the possibility of the discovery. We assume that the search mission is successful when both main wreckage and most survivors are found.
\\We run the program on our plan of searching and a reference plan (in which numbers of each plane are the same and planes are assigned to each zone uniformly). The results are shown in Figure \ref{cpp}. We put the model through 1000 times of simulation. Our plan obtains \textbf{154 times} of success and the reference plan obtains \textbf{52 times}. Clearly, our plan is more \textit{effective}.
\begin{figure}[H]
	\centering
	\includegraphics[width = 6.2in]{simsim2.jpg}
	\caption{Simulation results: the left one adopts our search model, while the right one adopts the reference plan}\label{cpp}
\end{figure}





\section{Conclusion}\label{Conclusion}
We are asked to build a generic mathematical model to use different types of search planes to search for a lost plane on open water. We first construct a model that will predict the position where the plane hits the surface of the ocean. Next, we build another model to determine the location of the main wreckage and the black box underwater. Then, we come up with 5 different search plans and a searching algorithm to guide the search operation. Finally, we test our model by simulating a practical air accident, and conclude that our models work much more efficient than ordinary ones. Therefore, we are confident that the models we build will indeed help improve the effectiveness of the search for a lost airplane.

\subsection{Strengths of the Models}
\begin{itemize}
	\item Model I: Our model is based on strict physical and mathematical analysis, including  kinematics and mechanics. Thus the calculation results are reliable.  We obtained several important parameters via professional simulations, which also contributes to the accuracy.
	
	\item Model II: The model is also based on strict physical and mathematical analysis. And we have considered each and every scenarios possible.
	
	\item Model III: Our model will provide proper solutions to any situations met in the search process. The searching priorities are quantized into specific values. The plans are detailed  and effective, and proves suitable for the airlines.
\end{itemize}

\subsection{Weaknesses of the Models}
\begin{itemize}
	\item Model I: Due to the limitations of the simulation software, it's difficult to precisely characterize the drag coefficient.
	
	\item Model II: It is difficult to characterize the effect of ocean current on the motion of the main wreckage and the black box. 
	
	\item Model III: The model neglects the time and effort consumed between shifts. 
	
\end{itemize}

\section{Error Analysis \& Limitations }\label{Error & limitations}

\subsection{Error Analysis}\label{error}

\subsubsection*{Error of $\rho_\text{air}$}
Recall the linear fit in \ref{Terms def I} for the density of air in troposphere:
\begin{figure}[H]
	\centering
	\includegraphics[width = 5.0in]{linearfittorho.jpg}
\end{figure}
\noindent We applied a linear fit to the data points \noindent  $\rho_\text{air} = \gamma h + \beta$, and it yields that the slope $\gamma = -8.0593\times 10^{-5} $ and the intercept $\beta = 1.16661 $ with standard errors $u_\gamma = 8.55052\times 10^{-8} $ and $u_\beta = 4.93701\times 10^{-4}$ and $R^2 = 0.98887$, which is very near to $1$.
\\Therefore, $\rho$ and $h$ are in a perfect linear relation.

\subsubsection*{Error of $C_l$}
Recall the linear fit in \ref{Terms def ii} for the lift coefficient:
\begin{figure}[H]
	\centering
	\includegraphics[width = 2.9in]{b707linearfit.jpg}
	\includegraphics[width = 2.9in]{b737linearfit.jpg}
\end{figure}
\noindent For the left set of data points: we applied a linear fit to the data points  $C_l = \xi \theta + \lambda$, and it yields that the slope $\xi = 0.11155 $ and the intercept $\lambda = -0.01108 $ with standard errors $u_\xi = 7.58411\times 10^{-4} $ and $u_\lambda = 0.00909$ and $R^2 = 0.99912$, which is very near to $1$.
\\Therefore, $C_l$ and $\theta$ are in a perfect linear relation.
\\~
\\For the right set of data points: we applied a linear fit to the data points  $C_l = \sigma \theta + \epsilon$, and it yields that the slope $\sigma = 0.10633$ and the intercept $\epsilon = 0.16701 $ with standard errors $u_\sigma = 0.00242 $ and $u_\epsilon = 0.01361$ and $R^2 = 0.99588$, which is very near to $1$.
\\Therefore, $C_l$ and $\theta$ are in a perfect linear relation.

\subsection{Limitations}
\begin{enumerate}
	\item The inaccurate estimation of the drag force.
	\\Our first and second model  both simplify the drag force to make the calculation more conveniently. But there are also some limitations. In the \textit{laminar region}, which the speed of fluid is slow, the drag force should be directly proportional to speed. However, in the \textit{turbulent region}, which the speed of fluid is much quicker, the drag force should be directly proportional to the \textbf{square of speed}. Our models assume that the whole process is in the \textit{laminar region}, thus the drag force is not accurate enough.
	
	\item Neglect of the capability of an excellent pilot.
	\\In \textbf{Model I}, we assumed that the angle of attack $\theta$ doesn't change during the whole descending process. However, in order to enhance the lift force and reduce the perpendicular velocity, a well-experienced pilot can change the angle of attack $\theta$ when needed. But this process is too difficult for us to simulate and calculate. Thus we just fix the angle of attack $\theta$, which will introduce small errors.
	
	\item Economics concerns.
	\\Our models assume that the airlines have enough planes to carry out the search operation.  However, if several planes are occupied to do the search work for nearly a month without any reward, the finance chain of this airline may be broken. Also we didn't consider the cost for the equipment in the search operation. Last but not least, the frequent disassembling  and installing of the equipment will inevitably be a large  cost to the airlines. 
\end{enumerate}




\section{Memo to the Managers of Airlines}\label{Memo}
Dear Manager of Airlines:

In order to save people's lives as soon as possible and find out the causes of the air accident, our team presents a plan to optimize and expedite the search for a lost plane.

Our plan consists of two main parts: predicting the crash location and arranging limited resources to search. To make our prediction more precise, our team adopt a modified mathematical model, which involves both the lift force and the drag force rather than a simple projectile motion in the air. Despite all the unexpected factors in the air, we manage to limit the possible crash position in a relatively small area, which will greatly reduce the cost of a follow-up  search operation. The region to be searched is a circle centered at the possible crash position we predicted.

When it comes the main wreckage and the black box, we have another model that simulates their motion underwater. Taking the pushing effect of the ocean current and the drag force underwater into consideration, we could also predict the location of the main wreckage and the black box on the seabed.

After determining the search region on the surface of the ocean and the location of wreckage underwater, we now could carry out the search operation. To optimize the search operation, we constructed 5 different search plans for 5 specific situations. We have a regular search plan to begin with, and different plans when we discovered survivors, floating wreckage, oil spills or signs of disintegration and explosion in the air.  We also have a searching algorithm to guide the usage of the 5 plans at our disposal.

The search operation involves 3 different sorts of planes, which focus on survivors, floating wreckage, and the main wreckage underwater, respectively. By reinstalling on-board equipment, the functions of the planes can vary. We also would like to introduce helicopters, ships, and submarines to help with the rescue and salvage. And it would be advantageous for pilots to know that the ocean current will carry the floating wreckage and the survivors with it, thus the search zones are not fixed.

During the first few days of the search operation, our algorithm mainly focuses on the search and rescue for survivors. An appreciable number of planes will be assigned to zones that lies along the possible crashing direction. Once survivors are spotted in a certain zone, more planes will be re-directed there to expedite the search. When floating wreckage or oil spill is found, we will try to estimate the possible crashing position and rearrange the searching priority. When the floating wreckage indicates signs of disintegration or explosion in the air, we will execute all-out search in that certain area. As the time goes by, we will lay more emphasis on the search and salvage of wreckage rather than survivors.

To prove the effectiveness of our plan, we ran many simulations on our computers. The results of the simulations are promising. Therefore, we believe that the adoption of our methods for search and rescue will be of great value and help to you and to the families and friends of the lost ones.
\\~

Yours Sincerely
\\~

Team \# 33804





\begin{thebibliography}{99}\label{References}
	\bibitem{plane database} ``Aircraft Technical Data \& Specifications.'' \textit{Airliners.net}. Demand Media, Inc. 30 Jan. 2015. Web. 8 Feb 2015. \url{http://www.airliners.net/aircraft-data/}
	
	\bibitem{af447} Beekman, Daniel. ``Air France Flight 447 last passenger jet to vanish before Malaysia Airlines plane.'' {\it Nydailynews.com.} N.p, 9 Mar. 2014. Web. 8 Feb 2015. \url{http://www.nydailynews.com/news/world/air-france-flight-plane-vanish-malaysia-airlines-article-1.1716146}
	
	\bibitem{google} {\it Maps.google.com.} Google. 9 Feb. 2015. Web. 9 Feb 2015. \url{http://maps.google.com}
	
	\bibitem{ocean motion} ``Ocean Motion.'' {\it Oceanmotion.org.} NASA. 26 Jan 2015. Web. 8 Feb 2015. \url{http://oceanmotion.org/html/resources/oscar.htm}
	
	\bibitem{ocean surface currents} ``Ocean Surface Currents.'' {\it Oceancurrents.rsmas.miami.edu.} Gulf of Mexico Research Initiative. 12 Dec. 2013. Web. 9 Feb 2015. \url{http://oceancurrents.rsmas.miami.edu/index.html}
	
	\bibitem{rho} Shelquist, Richard. ``An Introduction to Air Density and Density Altitude Calculations.'' {\it Wahiduddin.net.} Shelquist Engineering, 30 Jan. 2015. Web. 8 Feb 2015. \url{http://wahiduddin.net/calc/density_altitude.htm}
	
	\bibitem{airfoil} ``UIUC Airfoil Coordinates Database.'' {\it M-selig.ae.illinois.edu}. UIUC Applied Aerodynamics Group, 27 Jan. 2015. Web. 8 Feb 2015. \url{http://m-selig.ae.illinois.edu/ads/coord_database.html}
		 
\end{thebibliography}



\end{spacing}
\end{document}