\documentclass[10pt]{beamer}
\usetheme{default}
\usepackage{float}
\usepackage{animate}

\usepackage{esvect}
%\usepackage[T1]{fontenc}

%IndentfFirst 
%\usepackage{indentfirst}
%\setlength{\parindent}{2em}
%\setlength{\parskip}{2em}


%Geometry
\usepackage{geometry}
\geometry{left = 0.25in,right = 0.25in}



%Adjust frametitle position (height,lateral position, etc)
\defbeamertemplate*{frametitle}{smoothbars theme}
{%
	%\nointerlineskip%
	\begin{beamercolorbox}[wd=\paperwidth,leftskip=.5cm,rightskip=.3cm plus1fil,vmode]{frametitle}
		\vskip +4.5ex
		\usebeamerfont*{frametitle}\insertframetitle%
		\vskip -0.9ex
	\end{beamercolorbox}%
}


\begin{document}


%headline

\setbeamertemplate{headline}{
\parbox{\linewidth}{\vspace*{8pt}\centering{ 
		    {\color{blue!40!black}\insertsection}
		}
	}
}


%footline
\setbeamertemplate{footline}[text line]{%
	\color{blue!40!black}\parbox{\linewidth}{\vspace*{-8pt}Michael Liu ~ (\insertshortinstitute)\hfill\insertshorttitle\hfill\insertshortdate~~~~~~\insertframenumber{}~/~\inserttotalframenumber}}

%Remove navigation symbols
\setbeamertemplate{navigation symbols}{}

\setbeamertemplate{frametitle continuation}[from second] 

\newcommand{\tabincell}[2]{\begin{tabular}{@{}#1@{}}#2\end{tabular}}

%\let\oldframe\frame\renewcommand\frame[1][allowframebreaks]{\oldframe[#1]}

%Title page	
\title[Vv255 Applied Calculus III]{Vv255 Applied Calculus III\\{\small Recitation IV}}   
\author[Michael Liu]{LIU Xieyang\\{\tiny Teaching Assistant}} 
\institute[UM-SJTU JI]{University of Michigan - Shanghai Jiaotong University \\Joint Institute}
\date[Summer 2015]{Summer Term 2015} 
\begin{frame}
	\titlepage
\end{frame}

%Table of Contents (All)
%\begin{frame}
%	\frametitle{Table of Contents}
%	\tableofcontents
%\end{frame}

%Table of contents (before, highlight each section)
\AtBeginSection[]{
	\begin{frame}
		\frametitle{Table of Contents}
		\frametitle{Contents}
		\tableofcontents[currentsection]
	\end{frame}}




%section
\section{Lecture 8: Functions of several variables, Limits and Continuity} 



\begin{frame}[allowframebreaks]{Functions of several variables}
Definition:
\begin{itemize}
	\item A function of {\color{red}two} variables, is a rule that assigns a unique real number $f (x, y)$
	to each point $(x, y)$ in some set $D$ in the $xy$-plane.
	\item A function of {\color{red}three} variables, is a rule that assigns a unique real number
	$f (x, y, z)$ to each point $(x, y, z)$ in some set $D$ in three-dimensional space.
	\item {\color{blue}Natural Domain}: the
	domain of independent variables consisting of all points for which the function formula yields a {\color{red}real} value for the
	dependent variable.
\end{itemize}
\noindent 
Question:
\\~
\\{\color{purple} Find and sketch the domain of the function
$$f(x,y) = \ln (9 - x^2 - 9y^2)$$
}


\end{frame}


\begin{frame}[allowframebreaks]{Limit}
	\begin{figure}[H]
		\centering
		\includegraphics[scale = 0.8]{2limit}\\~\\
		\includegraphics[scale = 0.8]{3limit}
	\end{figure}
	The first thing we can do and need to 	do in 3D is to {\color{red}specify the curve along which} we have 	consider the limit as $(x, y)$ approaches $(x_0, y_0)$.
	\newpage
$$$$
	Limit along a curve:
	\begin{figure}[H]
		\centering

		\includegraphics[scale = 0.43]{limitdef}
	\end{figure}
	\newpage
	General definition of limit:
	\begin{figure}[H]
		\centering
		
		\includegraphics[scale = 0.43]{limitdef2}
	\end{figure}
	The general definition of limit says nothing about along which curve should we approach a point, and we should expect that the limits {\color{red}along any smooth curve} $C$ are {\color{red}equal}.
	\\~\\The relationship between the limit along a specific curve and the general definition of limit are stated as follows:
	\begin{figure}[H]
		\centering
		\includegraphics[scale = 0.43]{limitjudge}
	\end{figure}
	\newpage
	{\color{purple}
	Question: 
	\\Find out the limit if it exists, or explain why it doesn't exist.
	$$\lim_{(x,y)\rightarrow (0,0)}\dfrac{6x^3y}{2x^4 + y^4}$$
}
\end{frame}


\begin{frame}[allowframebreaks]{Continuity}
Definition:

$$\lim_{(x,y)\rightarrow(a,b)} f(x,y)= f(a,b)$$
Basically, we require that {\color{red} the limit of the function
	and the value of the function to be the same at the point}.
\\Theorem:
\begin{figure}[H]
	\centering
	\includegraphics[scale = 0.43]{conthem}
\end{figure}
{\color{purple}
	Question: 
	\\Find out the limit if it exists, or explain why it doesn't exist.
	$$1. \lim_{(x,y)\rightarrow(6,3)}xy\cos(x-2y)$$
	$$2. \lim_{(x,y)\rightarrow (0,0)}\dfrac{xy}{\sqrt{x^2 + y^2}} 	~~~\text{Hint: {\it Use SQUEEZE THEOREM}}$$

}

\end{frame}





%section
\section{Lecture 9: Partial Derivatives} 



\begin{frame}[allowframebreaks]{Partial Derivative}
\begin{figure}[H]
	\centering
	\includegraphics[scale = 0.60]{parpic}
	\caption{Holding the {\color{red}y-value} constant.}
\end{figure}
\noindent Definition:
$$\left.\dfrac{\partial f}{\partial x}\right\vert_{(x_0,y_0)} = \lim_{h\rightarrow 0}\dfrac{f(x_0 + h, y_0) - f(x_0, y_0)}{h} = \left.\dfrac{d}{dx}f(x,y_0)\right\vert_{x = x_0}$$
$$\left.\dfrac{\partial f}{\partial y}\right\vert_{(x_0,y_0)} = \lim_{h\rightarrow 0}\dfrac{f(x_0, y_0 + h) - f(x_0, y_0)}{h} =  \left.\dfrac{d}{dy}f(x_0,y)\right\vert_{y = y_0} $$
\newpage

For functions of more than two independent variables, the definitions of partial derivatives are very similar. They are ordinary derivatives with respect to {\color{red} one independent variable} taken while the other
independent variables are {\color{red}held constant}.
\\~\\
{\color{purple}
	Question: 
	\\Find the indicated partial derivatives. $f(x,y) = \sqrt{x^2 + y^2}$. What's $f_x(3,4)$?
	
}
\end{frame}



\begin{frame}[allowframebreaks]{Implicit Differentiation}
Assume that the following equation $$f(x,y,z) = 0$$ defines $z$ as a function of the two independent variables $x$ and $y$ and the partial derivatives exists. 
\\We find $\dfrac{\partial z}{\partial x}$ by differentiate both sides of the equation w.r.t $x$ while holding $y$ constant and treating  $z$ as a differentiable function of $x$, say $z = g(x)$
\\~\\
{\color{purple}
	Question: 
	\\Implicitly differentiate on $\dfrac{\partial z}{\partial x}$ and $\dfrac{\partial z}{\partial y}$. $$x^2 + y^2 + z^2 = 3xyz$$
	
}
\end{frame}






\begin{frame}[allowframebreaks]{Higher-order Partial Derivatives}
	Second-order Partial Derivatives:
	\begin{itemize}
		\item If we differentiate a function f (x; y) twice, we obtain its second-order partial
		derivatives.
		\item For example:
		$$\dfrac{\partial^2f}{\partial x^2} = \dfrac{\partial}{\partial x}\left(\dfrac{\partial f}{\partial x}\right) = f_{xx}  = (f_x)_x~~~~~ \dfrac{\partial^2 f}{\partial x \partial y} = \dfrac{\partial}{\partial x}\left(\dfrac{\partial f}{\partial y}\right) = f_{yx} = (f_y)_x$$
	\end{itemize}
	
	Clairaut's Theorem:
	\begin{itemize}
		\item If $f(x, y)$ and its partial derivatives $f_x$ , $f_y$ , $f_{xy}$ and $f_{yx}$ are defined throughout an
		open region containing a point $(a, b)$ and are all continuous at $(a, b)$, then
		$$f_{xy} (a, b) = f_{yx} (a, b)$$
	\end{itemize}
	
	Higher-order Partial Derivatives:
	\begin{itemize}
		\item Basically the same. Mind the {\color{red}order}!
	\end{itemize}
	
\end{frame}

\end{document}