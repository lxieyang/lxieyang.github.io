\documentclass[10pt]{beamer}
\usetheme{default}
\usepackage{float}
\usepackage{animate}
\usepackage{amsmath}

\usepackage{esvect}
%\usepackage[T1]{fontenc}

%IndentfFirst 
%\usepackage{indentfirst}
%\setlength{\parindent}{2em}
%\setlength{\parskip}{2em}


%Geometry
\usepackage{geometry}
\geometry{left = 0.25in,right = 0.25in}



%Adjust frametitle position (height,lateral position, etc)
\defbeamertemplate*{frametitle}{smoothbars theme}
{%
	%\nointerlineskip%
	\begin{beamercolorbox}[wd=\paperwidth,leftskip=.5cm,rightskip=.3cm plus1fil,vmode]{frametitle}
		\vskip +4.5ex
		\usebeamerfont*{frametitle}\insertframetitle%
		\vskip -0.9ex
	\end{beamercolorbox}%
}


\begin{document}


%headline

\setbeamertemplate{headline}{
\parbox{\linewidth}{\vspace*{8pt}\centering{ 
		    {\color{blue!40!black}\insertsection}
		}
	}
}


%footline
\setbeamertemplate{footline}[text line]{%
	\color{blue!40!black}\parbox{\linewidth}{\vspace*{-8pt}Michael Liu ~ (\insertshortinstitute)\hfill\insertshorttitle\hfill\insertshortdate~~~~~~\insertframenumber{}~/~\inserttotalframenumber}}

%Remove navigation symbols
\setbeamertemplate{navigation symbols}{}

\setbeamertemplate{frametitle continuation}[from second] 

\newcommand{\tabincell}[2]{\begin{tabular}{@{}#1@{}}#2\end{tabular}}

%\let\oldframe\frame\renewcommand\frame[1][allowframebreaks]{\oldframe[#1]}

%Title page	
\title[Vv255 Applied Calculus III]{Vv255 Applied Calculus III\\{\small Recitation X}}   
\author[Michael Liu]{LIU Xieyang\\{\tiny Teaching Assistant}} 
\institute[UM-SJTU JI]{University of Michigan - Shanghai Jiaotong University \\Joint Institute}
\date[Summer 2015]{Summer Term 2015} 
\begin{frame}
	\titlepage
\end{frame}

%Table of Contents (All)
%\begin{frame}
%	\frametitle{Table of Contents}
%	\tableofcontents
%\end{frame}

%Table of contents (before, highlight each section)
\AtBeginSection[]{
	\begin{frame}
		\frametitle{Table of Contents}
		\frametitle{Contents}
		\tableofcontents[currentsection]
	\end{frame}}




%section
\section{Lecture 22: The Fundamental theorem for line integrals} 



\begin{frame}[allowframebreaks]{Region of Interest  $D$}
	There are three common adjectives in front of the region of interest $D$ in the field of line integral.
	\begin{enumerate}
		\item {\color{red}Open}: the points on all the boundaries does NOT count!$$\includegraphics[scale = 0.15]{open}$$
		\item {\color{red}Connected}: any two points in $D$ can
		actually be connected by a path that lies entirely within D.$$\includegraphics[scale = 0.15]{connected}$$
		\item {\color{red}Simply connected}: one piece + NO ``holes''. Or say, any closed path in the region could be shrunk to a point.$$\includegraphics[scale = 0.15]{simplyconnected}$$
	\end{enumerate}

\end{frame}



\begin{frame}[allowframebreaks]{The Fundamental theorem for line integrals}
	\begin{figure}[H]
		\includegraphics[scale = 0.35]{ftcline}
	\end{figure}
	{\color{blue} Independent of path:}\\
	- Conditions:\\
	\begin{itemize}
		\item $\mathbf{F}$ being conservative: $\mathbf{F} = \nabla f$
		\item Region $D$ being {\color{red}open}.
		\item $P(x,y)$ and $Q(x,y)$ being continuous.
		\item $C$ being piecewise smooth.
		\item $A$ and $B$ are in region $D$.
	\end{itemize}
	$$\int_C \mathbf{F}\cdot d\mathbf{r} = \int_C\nabla f\cdot d\mathbf{r} = f(\mathbf{r}(b)) - f(\mathbf{r}(a))$$
	- The FTL states the value of the integral depends {\color{red}on the endpoints} but not on the actual path $C$, it is said to be {\color{blue}independent of the path}.\\
	- The FTL can be easily extended to $\mathbb{R}^3$.
	$$\text{Conservative}\xrightarrow{D ~\text{being {\color{red}Open}}} \text{Independent of path}$$
\end{frame}


\begin{frame}[allowframebreaks]{The Fundamental theorem for line integrals}

{\color{blue}Conservative}:\\
If the line integral of a vector field F is {\color{red}independent of path} within $D$, then F is a
{\color{blue}conservative} vector field on $D$.
\\Proof see lecture.
	$$\text{Independent of path}\xrightarrow{D ~\text{being {\color{red}Open \& Connected}}} \text{Conservative}$$
	$$$$
Therefore, we have the following iff statement:
$$\includegraphics[scale = 0.36]{iff}$$
\end{frame}



\begin{frame}[allowframebreaks]{Conservative Field Test}
$$\includegraphics[scale = 0.34]{contest}$$
$$\dfrac{\partial P}{\partial y} = \dfrac{\partial Q}{\partial x} \xrightarrow{D ~\text{being {\color{red}Open \& Simply Connected}}} {\text{Conservative}}$$
For $\mathbb{R}^3$, we can mimic the way introduced by your Physics professor M.K:
\begin{enumerate}
	\item Irrotational? ($\nabla\times \mathbf{F} = 0$)
	\item Open simply connected?
\end{enumerate} 
Actually consistent with the previous result!
\end{frame}


\begin{frame}[allowframebreaks]{The general technique of finding potential function}
-It remains to find the function $f$ such that $\nabla f = \mathbf{F}$ for a given vector field $\mathbf{F}$
	that is known to be conservative before we can apply FTL.
$$\includegraphics[scale = 0.40]{tech}$$

\end{frame}

\begin{frame}[allowframebreaks]{Conclusion}
$$\includegraphics[scale = 0.37]{conclusion}$$
\end{frame}



\section{Lecture 23: Green Theorem}
\begin{frame}[allowframebreaks]{Green Theorem}
Green's Theorem gives the relationship between a line integral around a simple closed curve $C$ and a double integral over the plane region $D$ bounded by $C$ .\\
In stating Green's
Theorem we use the convention that the {\color{red}positive orientation} of a simple closed curve $C$
refers to a single {\color{blue}counterclockwise} traversal of $C$. Thus, if $C$ is given by the vector function $\mathbf{r}(t), a\leq t\leq b$, then the region $D$ is always on the left as the point $\mathbf{r}(t)$ traverses $C$.
$$\includegraphics[scale = 0.3]{pnorientation}$$
$$$$
{\color{blue}Green's Theorem}\\
If $C$ is a positively oriented, piecewise smooth, simple
closed curve that encloses a region $D$, and $P(x, y)$ and $Q(x, y)$ are functions that
have continuous first partial derivatives on some open set containing $D$, then
$$\oint_C\mathbf{F}\cdot d\mathbf{r} = \oint_C Pdx + Qdy = \iint_D\left(\dfrac{\partial Q}{\partial x} - \dfrac{\partial P}{\partial y}\right)dA$$
where $$\mathbf{F} = P\mathbf{a}_x + Q\mathbf{a}_y~~~~\text{and}~~~~d\mathbf{r} = dx\mathbf{a}_x + dy\mathbf{a}_y$$
Proof see lecture.
\end{frame}



\begin{frame}[allowframebreaks]{Applications of Green Theorem}
\begin{itemize}
	\item Double integral is easier to evaluate than line integral, use Green's theorem in the positive direction.
	\item Line integral is easier to evaluate than double integral, use Green's theorem in the reverse direction.
	\item Computing areas.\\
	Note that the area of a region $D$ 	is $\iint_D 1dA$, we wish to choose $P$ and $Q$ so that $$\dfrac{\partial Q}{\partial x} - \dfrac{\partial P}{\partial y}  = 1$$
	There are several possibilities:
	\[
	\left\{
	\begin{aligned}
	P(x,y) &= 0\\
	Q(x,y) &=x
	\end{aligned}
	\right.~~~~
		\left\{
		\begin{aligned}
		P(x,y) &= -y\\
		Q(x,y) &=0
		\end{aligned}
		\right.
		~~~~
			\left\{
			\begin{aligned}
			P(x,y) &= -\frac{1}{2}y\\
			Q(x,y) &=\frac{1}{2}x
			\end{aligned}
			\right.
	\]
	Then Green's Theorem gives the following formulas for the area of $D$ :
	$$A = \oint_Cxdy = -\oint_Cydx = \dfrac{1}{2} \oint_Cxdy - ydx$$
	For example, the area of a ellipse $\dfrac{x^2}{a^2} + \dfrac{y^2}{b^2} = 1$ can be computed in the following way:\\
	The ellipse has parametric equations: $x=a\cos t, y=b\sin t$, where $0\leq t\leq 2\pi$. Thus:
\[
\begin{aligned}
A &= \dfrac{1}{2}\int_C	xdy - ydx
\\&= \int_0^{2\pi}(a\cos t)(b\cos t)dt 
\\&~~~~~~- (b\sin t)(-a\sin t)dt
\\&=\dfrac{ab}{2}\int_0^{2\pi}dt =\pi ab
\end{aligned}
~~~~~~~~\text{or}~~~~~~
\begin{aligned}
A &= \oint_C	xdy
\\&= \int_0^{2\pi}(a\cos t)(b\cos t)dt
\\&=ab\int_0^{2\pi}\cos^2 tdt 
\\&=\pi ab
\end{aligned}
\]
\item Help you understand Stoke's Theorem! (Will be covered later!)
\item Physical meaning:
$$\includegraphics[scale = 0.36]{phy}$$
Green's theorem says that if you add up all the {\color{red}microscopic} circulation inside $C$,
then the sum is exactly the same as the {\color{blue}macroscopic} circulation around $C$.
\item Normal \& Tangential form
$$\includegraphics[scale = 0.34]{normaltangential}$$
where $\mathbf{n}$ is the unit outward normal, and the $\mathbf{T}$ is the unit tangent vector.
\end{itemize}

\end{frame}

\end{document}